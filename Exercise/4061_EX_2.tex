\documentclass[letterpaper]{article} 
\usepackage[utf8]{inputenc}
\usepackage[T1]{fontenc}
\usepackage{amsmath}
\usepackage{amsfonts}
\usepackage{amssymb}
\usepackage{array}
\usepackage{booktabs}
\usepackage{hyperref}
\usepackage[version=4]{mhchem}
\usepackage{stmaryrd}
\usepackage[dvipsnames]{xcolor}
\colorlet{LightRubineRed}{RubineRed!70}
\colorlet{Mycolor1}{green!10!orange}
\definecolor{Mycolor2}{HTML}{00F9DE}
\usepackage{graphicx}
\usepackage{amsmath}
\usepackage{graphicx}
\usepackage{capt-of}
\usepackage{lipsum}
\usepackage{fancyvrb}
\usepackage{tabularx}
\usepackage{listings}
\usepackage[export]{adjustbox}
\graphicspath{ {./images/} }
\usepackage[utf8]{inputenc}
\usepackage[english]{babel}
\usepackage{float}
\usepackage{lipsum}
\usepackage{graphicx}
\usepackage{float}
\usepackage[margin=0.7in]{geometry}
\usepackage{amsmath}
\usepackage{graphicx}
\usepackage{capt-of}
\usepackage{tcolorbox}
\usepackage{lipsum}
\usepackage{graphicx}
\usepackage{float}
\usepackage{listings}
\usepackage{hyperref} 
\usepackage{xcolor} % For custom colors
\lstset{
	language=Python,                % Choose the language (e.g., Python, C, R)
	basicstyle=\ttfamily\small, % Font size and type
	keywordstyle=\color{blue},  % Keywords color
	commentstyle=\color{gray},  % Comments color
	stringstyle=\color{red},    % String color
	numbers=left,               % Line numbers
	numberstyle=\tiny\color{gray}, % Line number style
	stepnumber=1,               % Numbering step
	breaklines=true,            % Auto line break
	backgroundcolor=\color{black!5}, % Light gray background
	frame=single,               % Frame around the code
}
\usepackage{float}
\usepackage[]{amsthm} %lets us use \begin{proof}
\usepackage[]{amssymb} %gives us the character \varnothing

	\title{Homework 5, MATH 4061}
	\author{Zongyi Liu}
	\date{Mon, Oct 13, 2025}
	
	\begin{document}
		\maketitle
		
		
		\noindent
		\textbf{1.} Prove that the empty set is a subset of every set.
		
		\noindent
		\textbf{2.} A complex number $z$ is said to be \textit{algebraic} if there are integers $a_0, \ldots, a_n$, not all zero, such that
		\[
		a_0 z^n + a_1 z^{n-1} + \cdots + a_{n-1}z + a_n = 0.
		\]
		Prove that the set of all algebraic numbers is countable. \textit{Hint:} For every positive integer $N$ there are only finitely many equations with
		\[
		n + |a_0| + |a_1| + \cdots + |a_n| = N.
		\]
		
		\noindent
		\textbf{3.} Prove that there exist real numbers which are not algebraic.
		
		\noindent
		\textbf{4.} Is the set of all irrational real numbers countable?
		
		\noindent
		\textbf{5.} Construct a bounded set of real numbers with exactly three limit points.
		
		\noindent
		\textbf{6.} Let $E'$ be the set of all limit points of a set $E$.  
		Prove that $E'$ is closed. Prove that $E$ and $\bar{E}$ have the same limit points.  
		(Recall that $\bar{E} = E \cup E'$.) Do $E$ and $E'$ always have the same limit points?
		
		\clearpage
		
		\noindent
		\textbf{7.} Let $A_1, A_2, A_3, \ldots$ be subsets of a metric space.
		\begin{enumerate}
			\item[(a)] If $B_n = \bigcup_{i=1}^n A_i$, prove that $\bar{B}_n = \bigcup_{i=1}^n \bar{A}_i$, for $n = 1, 2, 3, \ldots$.
			\item[(b)] If $B = \bigcup_{i=1}^\infty A_i$, prove that $\bar{B} \supset \bigcup_{i=1}^\infty \bar{A}_i$.  
			Show, by an example, that this inclusion can be proper.
		\end{enumerate}
		
		\noindent
		\textbf{8.} Is every point of every open set $E \subset \mathbb{R}^2$ a limit point of $E$?  
		Answer the same question for closed sets in $\mathbb{R}^2$.
		
		\noindent
		\textbf{9.} Let $E^\circ$ denote the set of all interior points of a set $E$.  
		[\textit{See Definition 2.18(e);} $E^\circ$ is called the \textit{interior} of $E$.]
		\begin{enumerate}
			\item[(a)] Prove that $E^\circ$ is always open.
			\item[(b)] Prove that $E$ is open if and only if $E^\circ = E$.
			\item[(c)] If $G \subset E$ and $G$ is open, prove that $G \subset E^\circ$.
			\item[(d)] Prove that the complement of $E^\circ$ is the closure of the complement of $E$.
			\item[(e)] Do $E$ and $\bar{E}$ always have the same interiors?
			\item[(f)] Do $E$ and $E'$ always have the same closures?
		\end{enumerate}
		
		
			
			\noindent
			\textbf{10.} Let $X$ be an infinite set. For $p \in X$ and $q \in X$, define
			\[
			d(p,q) =
			\begin{cases}
				1, & \text{if } p \ne q, \\
				0, & \text{if } p = q.
			\end{cases}
			\]
			Prove that this is a metric. Which subsets of the resulting metric space are open? Which are closed? Which are compact?
			
		
			\noindent
			\textbf{11.} For $x \in \mathbb{R}^1$ and $y \in \mathbb{R}^1$, define
			\[
			\begin{aligned}
				d_1(x,y) &= (x - y)^2, \\
				d_2(x,y) &= \sqrt{|x - y|}, \\
				d_3(x,y) &= |x^2 - y^2|, \\
				d_4(x,y) &= |x - 2y|, \\
				d_5(x,y) &= \frac{|x - y|}{1 + |x - y|}.
			\end{aligned}
			\]
			Determine, for each of these, whether it is a metric or not.
			
			
			\noindent
			\textbf{12.} Let $K \subset \mathbb{R}^1$ consist of $0$ and the numbers $1/n$, for $n = 1, 2, 3, \ldots$.  
			Prove that $K$ is compact directly from the definition (without using the Heine–Borel theorem).
			
			
			\noindent
			\textbf{13.} Construct a compact set of real numbers whose limit points form a countable set.
			
			
			\noindent
			\textbf{14.} Give an example of an open cover of the segment $(0,1)$ which has no finite subcover.
			
			
			\noindent
			\textbf{15.} Show that Theorem 2.36 and its Corollary become false (in $\mathbb{R}^1$, for example) if the word “compact” is replaced by “closed” or by “bounded.”
			
			
			\noindent
			\textbf{16.} Regard $\mathbb{Q}$, the set of all rational numbers, as a metric space, with $d(p,q) = |p - q|$.  
			Let $E$ be the set of all $p \in \mathbb{Q}$ such that $2 < p^2 < 3$.  
			Show that $E$ is closed and bounded in $\mathbb{Q}$, but that $E$ is not compact.  
			Is $E$ open in $\mathbb{Q}$?
			
			
			\noindent
			\textbf{17.} Let $E$ be the set of all $x \in [0,1]$ whose decimal expansion contains only the digits $4$ and $7$.  
			Is $E$ countable? Is $E$ dense in $[0,1]$? Is $E$ compact? Is $E$ perfect?
			
			
			\noindent
			\textbf{18.} Is there a nonempty perfect set in $\mathbb{R}^1$ which contains no rational number?
			
			
			\noindent
			\textbf{19.} (a) If $A$ and $B$ are disjoint closed sets in some metric space $X$, prove that they are separated.  
			(b) Prove the same for disjoint open sets.  
			(c) Fix $p \in X$, $\delta > 0$, define $A$ to be the set of all $q \in X$ for which $d(p,q) < \delta$, define $B$ similarly, with $>$ in place of $<$. Prove that $A$ and $B$ are separated.  
			(d) Prove that every connected metric space with at least two points is uncountable. \textit{Hint:} Use (c).
			
			
			\noindent
			\textbf{20.} Are closures and interiors of connected sets always connected? (Look at subsets of $\mathbb{R}^2$.)
			
			
			\noindent
			\textbf{21.} Let $A$ and $B$ be separated subsets of some $\mathbb{R}^k$, suppose $a \in A$, $b \in B$, and define
			\[
			p(t) = (1 - t)a + tb
			\]
			for $t \in \mathbb{R}^1$. Put $A_0 = p^{-1}(A)$, $B_0 = p^{-1}(B)$.  
			[Thus $t \in A_0$ if and only if $p(t) \in A$.]
		\end{document}
		
		
	