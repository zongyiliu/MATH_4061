\documentclass[letterpaper]{article} 
\usepackage[utf8]{inputenc}
\usepackage[T1]{fontenc}
\usepackage{amsmath}
\usepackage{amsfonts}
\usepackage{amssymb}
\usepackage{array}
\usepackage{booktabs}
\usepackage{hyperref}
\usepackage[version=4]{mhchem}
\usepackage{stmaryrd}
\usepackage[dvipsnames]{xcolor}
\colorlet{LightRubineRed}{RubineRed!70}
\colorlet{Mycolor1}{green!10!orange}
\definecolor{Mycolor2}{HTML}{00F9DE}
\usepackage{graphicx}
\usepackage{amsmath}
\usepackage{graphicx}
\usepackage{capt-of}
\usepackage{lipsum}
\usepackage{fancyvrb}
\usepackage{tabularx}
\usepackage{listings}
\usepackage[export]{adjustbox}
\graphicspath{ {./images/} }
\usepackage[utf8]{inputenc}
\usepackage[english]{babel}
\usepackage{float}
\usepackage{lipsum}
\usepackage{graphicx}
\usepackage{float}
\usepackage[margin=0.7in]{geometry}
\usepackage{amsmath}
\usepackage{graphicx}
\usepackage{capt-of}
\usepackage{tcolorbox}
\usepackage{lipsum}
\usepackage{graphicx}
\usepackage{float}
\usepackage{listings}
\usepackage{hyperref} 
\usepackage{xcolor} % For custom colors
\lstset{
	language=Python,                % Choose the language (e.g., Python, C, R)
	basicstyle=\ttfamily\small, % Font size and type
	keywordstyle=\color{blue},  % Keywords color
	commentstyle=\color{gray},  % Comments color
	stringstyle=\color{red},    % String color
	numbers=left,               % Line numbers
	numberstyle=\tiny\color{gray}, % Line number style
	stepnumber=1,               % Numbering step
	breaklines=true,            % Auto line break
	backgroundcolor=\color{black!5}, % Light gray background
	frame=single,               % Frame around the code
}
\usepackage{float}
\usepackage[]{amsthm} %lets us use \begin{proof}
\usepackage[]{amssymb} %gives us the character \varnothing

	\title{Exercise 2, MATH 4061}
	\author{Zongyi Liu}
	\date{Mon, Oct 13, 2025}
	
	\begin{document}
		\maketitle
		\section{Problem 1} 
		
		Prove that the empty set is a subset of every set.
		
		\textbf{Answer}
		
		\clearpage
		
		\section{Problem 2} 
		A complex number $z$ is said to be \textit{algebraic} if there are integers $a_0, \ldots, a_n$, not all zero, such that
		\[
		a_0 z^n + a_1 z^{n-1} + \cdots + a_{n-1}z + a_n = 0.
		\]
		Prove that the set of all algebraic numbers is countable. \textit{Hint}: For every positive integer $N$ there are only finitely many equations with
		\[
		n + |a_0| + |a_1| + \cdots + |a_n| = N.
		\]
		
		
		\textbf{Answer}
		
		\clearpage
		
		\section{Problem 3} 
		Prove that there exist real numbers which are not algebraic.
		
		\textbf{Answer}
		
		\clearpage
		
		\section{Problem 4}
		Is the set of all irrational real numbers countable?
		
		\textbf{Answer}
		
		\clearpage
		
		\section{Problem 5} 
		Construct a bounded set of real numbers with exactly three limit points.
		
		\textbf{Answer}
		
		\clearpage
		
		\section{Problem 6}
		
		Let $E'$ be the set of all limit points of a set $E$.  
		Prove that $E'$ is closed. Prove that $E$ and $\bar{E}$ have the same limit points. (Recall that $\bar{E} = E \cup E'$.) Do $E$ and $E'$ always have the same limit points?
		
		\textbf{Answer}
		
		\clearpage
		
		\section{Problem 7}
		Let $A_1, A_2, A_3, \ldots$ be subsets of a metric space.
		\begin{enumerate}
			\item[(a)] If $B_n = \bigcup_{i=1}^n A_i$, prove that $\bar{B}_n = \bigcup_{i=1}^n \bar{A}_i$, for $n = 1, 2, 3, \ldots$.
			\item[(b)] If $B = \bigcup_{i=1}^\infty A_i$, prove that $\bar{B} \supset \bigcup_{i=1}^\infty \bar{A}_i$.  
			Show, by an example, that this inclusion can be proper.
		\end{enumerate}
		
		\textbf{Answer}
		
		\clearpage
		
		\section{Problem 8}
		Is every point of every open set $E \subset \mathbb{R}^2$ a limit point of $E$? Answer the same question for closed sets in $\mathbb{R}^2$.
		
		\textbf{Answer}
		
		\clearpage
		
		\section{Problem 9}
		Let $E^\circ$ denote the set of all interior points of a set $E$.  
		[\textit{See Definition 2.18(e);} $E^\circ$ is called the \textit{interior} of $E$.]
		\begin{enumerate}
			\item[(a)] Prove that $E^\circ$ is always open.
			\item[(b)] Prove that $E$ is open if and only if $E^\circ = E$.
			\item[(c)] If $G \subset E$ and $G$ is open, prove that $G \subset E^\circ$.
			\item[(d)] Prove that the complement of $E^\circ$ is the closure of the complement of $E$.
			\item[(e)] Do $E$ and $\bar{E}$ always have the same interiors?
			\item[(f)] Do $E$ and $E'$ always have the same closures?
		\end{enumerate}
		
		\textbf{Answer}
		
		\clearpage
		
		\section{Problem 10} 
		Let $X$ be an infinite set. For $p \in X$ and $q \in X$, define
		\[
		d(p,q) =
		\begin{cases}
			1, & \text{if } p \ne q, \\
			0, & \text{if } p = q.
		\end{cases}
		\]
		Prove that this is a metric. Which subsets of the resulting metric space are open? Which are closed? Which are compact?
		
		
		\textbf{Answer}
		
		\clearpage
		
		\section{Problem 11} 
		For $x \in \mathbb{R}^1$ and $y \in \mathbb{R}^1$, define
		\[
		\begin{aligned}
			d_1(x,y) &= (x - y)^2, \\
			d_2(x,y) &= \sqrt{|x - y|}, \\
			d_3(x,y) &= |x^2 - y^2|, \\
			d_4(x,y) &= |x - 2y|, \\
			d_5(x,y) &= \frac{|x - y|}{1 + |x - y|}.
		\end{aligned}
		\]
		Determine, for each of these, whether it is a metric or not.
		
		
		\textbf{Answer}
		
		\clearpage
		
		\section{Problem 12} 
		Let $K \subset \mathbb{R}^1$ consist of $0$ and the numbers $1/n$, for $n = 1, 2, 3, \ldots$.  
		Prove that $K$ is compact directly from the definition (without using the Heine–Borel theorem).
		
		
		\textbf{Answer}
		
		\clearpage
		
		\section{Problem 13} 
		Construct a compact set of real numbers whose limit points form a countable set.
		
		\textbf{Answer}
		
		\clearpage
		
		\section{Problem 14} 
		Give an example of an open cover of the segment $(0,1)$ which has no finite subcover.
		
		
		\textbf{Answer}
		
		\clearpage
		
		\section{Problem 15} 
		Show that Theorem 2.36 and its Corollary become false (in $\mathbb{R}^1$, for example) if the word “compact” is replaced by “closed” or by “bounded.”
		
		
		\textbf{Answer}
		
		\clearpage
		
		\section{Problem 16} 
		Regard $\mathbb{Q}$, the set of all rational numbers, as a metric space, with $d(p,q) = |p - q|$.  
		Let $E$ be the set of all $p \in \mathbb{Q}$ such that $2 < p^2 < 3$.  
		Show that $E$ is closed and bounded in $\mathbb{Q}$, but that $E$ is not compact.  
		Is $E$ open in $\mathbb{Q}$?
		
		\textbf{Answer}
		
		\clearpage
		
		\section{Problem 17} 
		Regard $\mathbb{Q}$, the set of all rational numbers, as a metric space, with $d(p,q) = |p - q|$.  
		Let $E$ be the set of all $p \in \mathbb{Q}$ such that $2 < p^2 < 3$.  
		Show that $E$ is closed and bounded in $\mathbb{Q}$, but that $E$ is not compact.  
		Is $E$ open in $\mathbb{Q}$?
		
		\textbf{Answer}
		
		\clearpage
		
		\section{Problem 18} 
		Is there a nonempty perfect set in $\mathbb{R}^1$ which contains no rational number?
		
		
		\textbf{Answer}
		
		\clearpage
		
		\section{Problem 19} 
		(a) If $A$ and $B$ are disjoint closed sets in some metric space $X$, prove that they are separated.  
		
		(b) Prove the same for disjoint open sets. 
		
		(c) Fix $p \in X$, $\delta > 0$, define $A$ to be the set of all $q \in X$ for which $d(p,q) < \delta$, define $B$ similarly, with $>$ in place of $<$. Prove that $A$ and $B$ are separated.
		
		(d) Prove that every connected metric space with at least two points is uncountable. \textit{Hint}: Use (c).
		
		
		\textbf{Answer}
		
		\clearpage
		
		\section{Problem 20} 
		Are closures and interiors of connected sets always connected? (Look at subsets of $\mathbb{R}^2$.)
		
		
		\textbf{Answer}
		
		\clearpage
		
		\section{Problem 21} 
		Let $A$ and $B$ be separated subsets of some $\mathbb{R}^k$, suppose $a \in A$, $b \in B$, and define
		\[
		p(t) = (1 - t)a + tb
		\]
		for $t \in \mathbb{R}^1$. Put $A_0 = p^{-1}(A)$, $B_0 = p^{-1}(B)$.  
		[Thus $t \in A_0$ if and only if $p(t) \in A$.]
		
		(a)\ Prove that $A_0$ and $B_0$ are separated subsets of $\mathbb{R}^1$.
		
		 (b)\ Prove that there exists $t_0 \in (0,1)$ such that $\rho(t_0) \notin A \cup B$.
		
		 (c)\ Prove that every convex subset of $\mathbb{R}^k$ is connected.
		
		\textbf{Answer}
		
		\clearpage
		
		\section{Problem 22} 
		A metric space is called \emph{separable} if it contains a countable dense subset.
		Show that $\mathbb{R}^k$ is separable.
		\textit{Hint:} Consider the set of points whose coordinates are rational.
		
		
		\textbf{Answer}
		
		\clearpage
		
		\section{Problem 23} 
		A collection $\{V_\alpha\}$ of open subsets of $X$ is said to be a \emph{base} for $X$
		if the following is true: For every $x \in X$ and every open set $G \subset X$ such that $x \in G$,
		there exists an $\alpha$ such that $x \in V_\alpha \subset G$.
		In other words, every open set in $X$ is the union of a subcollection of $\{V_\alpha\}$.
		
		Prove that every separable metric space has a countable base.
		\textit{Hint:} Take all neighborhoods with rational radius and center in some countable dense subset of $X$.
		
		
		\textbf{Answer}
		
		\clearpage
		
		\section{Problem 24} 
		Let $X$ be a metric space in which every infinite subset has a limit point. Prove that $X$ is separable. \textit{Hint:} Fix $\delta > 0$, and pick $x_1 \in X$. Having chosen $x_1,\dots,x_j \in X$,
		choose $x_{j+1} \in X$, if possible, so that $d(x_i, x_{j+1}) \ge \delta$ for $i=1,\dots,j$. Show that this process must stop after finitely many steps, and that $X$ can therefore be covered by finitely many neighborhoods of radius $\delta$. Take $\delta = 1/n$ ($n = 1,2,3,\dots$), and consider the centers of the corresponding neighborhoods.
		
		
		\textbf{Answer}
		
		\clearpage
		
		\section{Problem 25} 
		Prove that every compact metric space $K$ has a countable base, and that $K$ is therefore separable. \textit{Hint:} For every positive integer $n$, there are finitely many neighborhoods of radius $1/n$
		whose union covers $K$.
		
		\textbf{Answer}
		
		\clearpage
		
		\section{Problem 26} 
		Let $X$ be a metric space in which every infinite subset has a limit point. Prove that $X$ is compact. \textit{Hint:} By Exercises 23 and 24, $X$ has a countable base. It follows that every open cover
		of $X$ has a countable subcover $\{G_n\}$ ($n=1,2,3,\dots$).
		If no finite subcollection of $\{G_n\}$ covers $X$, then the complement $F_n$ of $G_1 \cup \cdots \cup G_n$ is nonempty for each $n$, but $\bigcap_n F_n$ is empty.	If $E$ is a set which contains a point from each $F_n$, consider a limit point of $E$,
		and obtain a contradiction.
		
		\textbf{Answer}
		
		\clearpage
		
		\section{Problem 27} 
		Define a point $p$ in a metric space $X$ to be a \emph{condensation point} of a set $E \subset X$
		if every neighborhood of $p$ contains uncountably many points of $E$.
		
		Suppose $E \subset \mathbb{R}^k$, $E$ uncountable, and let $P$ be the set of all condensation points of $E$.
		Prove that $P$ is perfect and that at most countably many points of $E$ are not in $P$.
		
		If $\{V_n\}$ is a countable base of $\mathbb{R}^k$, let $W$ be the union of those $V_n$
		for which $E \cap V_n$ is at most countable, and show that $P = W^c$.
		
		\textbf{Answer}
		
		\clearpage
		
		\section{Problem 28} 
		Prove that every closed set in a separable metric space is the union of a (possibly empty) perfect set and a set which is at most countable. *\textit{Corollary: Every countable closed set in $\mathbb{R}^k$ has isolated points.}
		\textit{Hint:} Use Exercise 27.
		
		\textbf{Answer}
		
		\clearpage
		
		\section{Problem 29} 
		Prove that every open set in $\mathbb{R}^1$ is the union of an at most countable
		collection of disjoint segments.
		\textit{Hint:} Use Exercise 22.
		
		\textbf{Answer}
		
		\clearpage
		
		\section{Problem 30}
		
		Imitate the proof of Theorem~2.43 to obtain the following result:
		
		\[
		\text{If } \mathbb{R}^k = \bigcup_{i=1}^{\infty} F_n,\ \text{where each $F_n$ is a closed subset of } \mathbb{R}^k,
		\text{ then at least one $F_n$ has a nonempty interior.}
		\]
		
		\textit{Equivalent statement:} If $G_n$ is a dense open subset of $\mathbb{R}^k$,
		for $n=1,2,3,\dots$, then
		
		\[
		\bigcap_{i=1}^{\infty} G_n\ \text{ is not empty (in fact, it is dense in } \mathbb{R}^k).
		\]
		
		(This is a special case of Baire’s theorem; see Exercise 22, Chap.~3, for the general case.)
		
		\textbf{Answer}
		
		\clearpage
		\end{document}
		
		
	
