\documentclass[letterpaper]{article} 
\usepackage[utf8]{inputenc}
\usepackage[T1]{fontenc}
\usepackage{amsmath}
\usepackage{amsfonts}
\usepackage{amssymb}
\usepackage{array}
\usepackage{booktabs}
\usepackage{hyperref}
\usepackage[version=4]{mhchem}
\usepackage{stmaryrd}
\usepackage[dvipsnames]{xcolor}
\colorlet{LightRubineRed}{RubineRed!70}
\colorlet{Mycolor1}{green!10!orange}
\definecolor{Mycolor2}{HTML}{00F9DE}
\usepackage{graphicx}
\usepackage{amsmath}
\usepackage{graphicx}
\usepackage{capt-of}
\usepackage{lipsum}
\usepackage{fancyvrb}
\usepackage{tabularx}
\usepackage{listings}
\usepackage[export]{adjustbox}
\graphicspath{ {./images/} }
\usepackage[utf8]{inputenc}
\usepackage[english]{babel}
\usepackage{float}
\usepackage{lipsum}
\usepackage{graphicx}
\usepackage{float}
\usepackage[margin=0.7in]{geometry}
\usepackage{amsmath}
\usepackage{graphicx}
\usepackage{capt-of}
\usepackage{tcolorbox}
\usepackage{lipsum}
\usepackage{graphicx}
\usepackage{float}
\usepackage{listings}
\usepackage{hyperref} 
\usepackage{xcolor} % For custom colors
\lstset{
	language=Python,                % Choose the language (e.g., Python, C, R)
	basicstyle=\ttfamily\small, % Font size and type
	keywordstyle=\color{blue},  % Keywords color
	commentstyle=\color{gray},  % Comments color
	stringstyle=\color{red},    % String color
	numbers=left,               % Line numbers
	numberstyle=\tiny\color{gray}, % Line number style
	stepnumber=1,               % Numbering step
	breaklines=true,            % Auto line break
	backgroundcolor=\color{black!5}, % Light gray background
	frame=single,               % Frame around the code
}
\usepackage{float}
\usepackage[]{amsthm} %lets us use \begin{proof}
\usepackage[]{amssymb} %gives us the character \varnothing
	\title{Exercise 4, MATH 4061}
\author{Zongyi Liu}
\date{Thu, Nov 27, 2025}

\begin{document}
	\maketitle
	
	
	\section{Problem 1}
	Suppose $f$ is a real function defined on $R^{1}$ which satisfies

$$
\lim _{h \rightarrow 0}[f(x+h)-f(x-h)]=0
$$

for every $x \in R^{1}$. Does this imply that $f$ is continuous?

\textbf{Answer}

\clearpage
	
	\section{Problem 2}
	
If $f$ is a continuous mapping of a metric space $X$ into a metric space $Y$, prove that

$$
f(\bar{E}) \subset \overline{f(E)}
$$

for every set $E \subset X$. ( $\bar{E}$ denotes the closure of $E$.) Show, by an example, that $f(\bar{E})$ can be a proper subset of $\overline{f(E)}$.

\textbf{Answer}

\clearpage

	\section{Problem 3} 
	
Let $f$ be a continuous real function on a metric space $X$. Let $Z(f)$ (the zero set of $f$) be the set of all $p \in X$ at which $f(p)=0$. Prove that $Z(f)$ is closed.

	\textbf{Answer}
	
	\clearpage
	
	\section{Problem 4}
	
	Let $f$ and $g$ be continuous mappings of a metric space $X$ into a metric space $Y$, and let $E$ be a dense subset of $X$. Prove that $f(E)$ is dense in $f(X)$. If $g(p)=f(p)$ for all $p \in E$, prove that $g(p)=f(p)$ for all $p \in X$. (In other words, a continuous mapping is determined by its values on a dense subset of its domain.)
	
	\textbf{Answer}
	
	\clearpage
	
	
	\section{Problem 5}
	
 If $f$ is a real continuous function defined on a closed set $E \subset R^{1}$, prove that there exist continuous real functions $g$ on $R^{1}$ such that $g(x)=f(x)$ for all $x \in E$. (Such functions $g$ are called continuous extensions of $f$ from $E$ to $R^{1}$.) Show that the result becomes false if the word "closed" is omitted. Extend the result to vectorvalued functions. 
 
 \emph{Hint}: Let the graph of $g$ be a straight line on each of the segments which constitute the complement of $E$ (compare Exercise 29, Chap. 2). The result remains true if $R^{1}$ is replaced by any metric space, but the proof is not so simple.
 
\textbf{Answer}

\clearpage

	
	\section{Problem 6} 
	If $f$ is defined on $E$, the graph of $f$ is the set of points $(x, f(x))$, for $x \in E$. In particular, if $E$ is a set of real numbers, and $f$ is real-valued, the graph of $f$ is a subset of the plane. 
	
	Suppose $E$ is compact, and prove that $f$ is continuous on $E$ if and only if its graph is compact.
	
	\textbf{Answer}
	
	\clearpage
	
	\section{Problem 7} 
	
	If $E \subset X$ and if $f$ is a function defined on $X$, the restriction of $f$ to $E$ is the function $g$ whose domain of definition is $E$, such that $g(p)=f(p)$ for $p \in E$. Define $f$ and $g$ on $R^{2}$ by: $f(0,0)=g(0,0)=0, f(x, y)=x y^{2} /\left(x^{2}+y^{4}\right), g(x, y)=x y^{2} /\left(x^{2}+y^{6}\right)$ if $(x, y) \neq(0,0)$. Prove that $f$ is bounded on $R^{2}$, that $g$ is unbounded in every neighborhood of $(0,0)$, and that $f$ is not continuous at $(0,0)$; nevertheless, the restrictions of both $f$ and $g$ to every straight line in $R^{2}$ are continuous!
	
	\textbf{Answer}
	
	\clearpage
	
	\section{Problem 8} 
	
	
	Let $f$ be a real uniformly continuous function on the bounded set $E$ in $R^{1}$. Prove that $f$ is bounded on $E$.
	
	Show that the conclusion is false if boundedness of $E$ is omitted from the hypothesis.
	
	\textbf{Answer}
	
	\clearpage
	
	\section{Problem 9}
	
Show that the requirement in the definition of uniform continuity can be rephrased as follows, in terms of diameters of sets: To every $\varepsilon>0$ there exists a $\delta>0$ such that diam $f(E)<\varepsilon$ for all $E \subset X$ with diam $E<\delta$.

\textbf{Answer}

\clearpage
	
	\section{Problem 10}
	
	Complete the details of the following alternative proof of Theorem 4.19: If $f$ is not uniformly continuous, then for some $\varepsilon>0$ there are sequences $\left\{p_{n}\right\},\left\{q_{n}\right\}$ in $X$ such that $d_{X}\left(p_{n}, q_{n}\right) \rightarrow 0$ but $d_{Y}\left(f\left(p_{n}\right), f\left(q_{n}\right)\right)>\varepsilon$. Use Theorem 2.37 to obtain a contradiction.
	
	\textbf{Answer}
	
	\clearpage
	
	\section{Problem 11}
	
	Suppose $f$ is a uniformly continuous mapping of a metric space $X$ into a metric space $Y$ and prove that $\left\{f\left(x_{n}\right)\right\}$ is a Cauchy sequence in $Y$ for every Cauchy sequence $\left\{x_{n}\right\}$ in $X$. Use this result to give an alternative proof of the theorem stated in Exercise 13.
	
	\textbf{Answer}
	
	\clearpage
	
	\section{Problem 12}
	
	A uniformly continuous function of a uniformly continuous function is uniformly continuous.
	
	State this more precisely and prove it.
	
	\textbf{Answer}
	
	\clearpage
	
	\section{Problem 13}
	
	Let $E$ be a dense subset of a metric space $X$, and let $f$ be a uniformly continuous real function defined on $E$. Prove that $f$ has a continuous extension from $E$ to $X$ (see Exercise 5 for terminology). (Uniqueness follows from Exercise 4.) Hint: For each $p \in X$ and each positive integer $n$, let $V_{n}(p)$ be the set of all $q \in E$ with $d(p, q)<1 / n$. Use Exercise 9 to show that the intersection of the closures of the sets $f\left(V_{1}(p)\right), f\left(V_{2}(p)\right), \ldots$, consists of a single point, say $g(p)$, of $R^{1}$. Prove that the function $g$ so defined on $X$ is the desired extension of $f$.
	
	Could the range space $R^{1}$ be replaced by $R^{k}$ ? By any compact metric space? By any complete metric space? By any metric space?
	
	\textbf{Answer}
	
	\clearpage
	
	\section{Problem 14}
	
Let $I=[0,1]$ be the closed unit interval. Suppose $f$ is a continuous mapping of $I$ into $I$. Prove that $f(x)=x$ for at least one $x \in I$.

\textbf{Answer}

\clearpage
	
	\section{Problem 15}
	
	Call a mapping of $X$ into $Y$ open if $f(V)$ is an open set in $Y$ whenever $V$ is an open set in $X$.
	
	Prove that every continuous open mapping of $R^{1}$ into $R^{1}$ is monotonic.
	
	\textbf{Answer}
	
	\clearpage
	
	\section{Problem 16}
	
	Let [$x$] denote the largest integer contained in $x$, that is, $[x]$ is the integer such that $x-1<[x] \leq x$; and let $(x)=x-[x]$ denote the fractional part of $x$. What discontinuities do the functions $[x]$ and $(x)$ have?
	
	\textbf{Answer}
	
	\clearpage
	
		
	\section{Problem 17}
	
	Let $f$ be a real function defined on $(a, b)$. Prove that the set of points at which $f$ has a simple discontinuity is at most countable. Hint: Let $E$ be the set on which $f(x-)<f(x+)$. With each point $x$ of $E$, associate a triple $(p, q, r)$ of rational numbers such that:
	
	(a) $f(x-)<p<f(x+)$,
	
	(b) $a<q<t<x$ implies $f(t)<p$,
	
	(c) $x<t<r<b$ implies $f(t)>p$.
	
	The set of all such triples is countable. Show that each triple is associated with at most one point of $E$. Deal similarly with the other possible types of simple discontinuities.
	
	\textbf{Answer}
	
	\clearpage
	
	
	\section{Problem 18}
	
	Every rational $x$ can be written in the form $x=m / n$, where $n>0$, and $m$ and $n$ are integers without any common divisors. When $x=0$, we take $n=1$. Consider the function $f$ defined on $R^{1}$ by
	
	Prove that $f$ is continuous at every irrational point, and that $f$ has a simple discontinuity at every rational point.
	
	\textbf{Answer}
	
	\clearpage
	
	\section{Problem 19} 
	
	Suppose $f$ is a real function with domain $R^{1}$ which has the intermediate value property: If $f(a)<c<f(b)$, then $f(x)=c$ for some $x$ between $a$ and $b$.
	
	Suppose also, for every rational $r$, that the set of all $x$ with $f(x)=r$ is closed. Prove that $f$ is continuous. 
	
	\emph{Hint}: If $x_{n} \rightarrow x_{0}$ but $f\left(x_{n}\right)>r>f\left(x_{0}\right)$ for some $r$ and all $n$, then $f\left(t_{n}\right)=r$ for some $t_{n}$ between $x_{0}$ and $x_{n}$; thus $t_{n} \rightarrow x_{0}$. Find a contradiction. (N. J. Fine, Amer. Math. Monthly, vol. 73, 1966, p. 782.)
	
	\textbf{Answer}
	
	\clearpage
	
	
	\section{Problem 20}
	
	If $E$ is a nonempty subset of a metric space $X$, define the distance from $x \in X$ to $E$ by
	
	$$
	\rho_{E}(x)=\inf _{z \in E} d(x, z) .
	$$
	
	(a) Prove that $\rho_{E}(x)=0$ if and only if $x \in \bar{E}$.
	
	(b) Prove that $\rho_{E}$ is a uniformly continuous function on $X$, by showing that
	
	$$
	\left|\rho_{E}(x)-\rho_{E}(y)\right| \leq d(x, y)
	$$
	
	for all $x \in X, y \in X$.
	
	\emph{Hint}: $\rho_{E}(x) \leq d(x, z) \leq d(x, y)+d(y, z)$, so that
	
	$$
	\rho_{E}(x) \leq d(x, y)+\rho_{E}(y)
	$$
	
	\textbf{Answer}
	
	\clearpage
	
	
	\section{Problem 21}
	
	Suppose $K$ and $F$ are disjoint sets in a metric space $X, K$ is compact, $F$ is closed. Prove that there exists $\delta>0$ such that $d(p, q)>\delta$ if $p \in K, q \in F$. Hint: $\rho_{F}$ is a continuous positive function on $K$.

Show that the conclusion may fail for two disjoint closed sets if neither is compact.

\textbf{Answer}

\clearpage
	
	\section{Problem 22}
	
	Let $A$ and $B$ be disjoint nonempty closed sets in a metric space $X$, and define
	
	$$
	f(p)=\frac{\rho_{A}(p)}{\rho_{A}(p)+\rho_{B}(p)} \quad(p \in X)
	$$
	
	Show that $f$ is a continuous function on $X$ whose range lies in $[0,1]$, that $f(p)=0$ precisely on $A$ and $f(p)=1$ precisely on $B$. This establishes a converse of Exercise 3: Every closed set $A \subset X$ is $Z(f)$ for some continuous real $f$ on $X$. Setting
	
	$$
	V=f^{-1}\left(\left[0, \frac{1}{2}\right)\right), \quad W=f^{-1}\left(\left(\frac{1}{2}, 1\right]\right),
	$$
	
	show that $V$ and $W$ are open and disjoint, and that $A \subset V, B \subset W$. (Thus pairs of disjoint closed sets in a metric space can be covered by pairs of disjoint open sets. This property of metric spaces is called normality.)
	
	\textbf{Answer}
	
	\clearpage
	
	\section{Problem 23}
	
	A real-valued function $f$ defined in $(a, b)$ is said to be convex if
	
	$$
	f(\lambda x+(1-\lambda) y) \leq \lambda f(x)+(1-\lambda) f(y)
	$$
	
	whenever $a<x<b, a<y<b, 0<\lambda<1$. Prove that every convex function is continuous. Prove that every increasing convex function of a convex function is convex. (For example, if $f$ is convex, so is $e^{f}$.)
	
	If $f$ is convex in ($a, b$) and if $a<s<t<u<b$, show that
	
	$$
	\frac{f(t)-f(s)}{t-s} \leq \frac{f(u)-f(s)}{u-s} \leq \frac{f(u)-f(t)}{u-t}
	$$
	
	\textbf{Answer}
	
	\clearpage
	
	\section{Problem 24}
	
Assume that $f$ is a continuous real function defined in ($a, b$) such that

$$
f\left(\frac{x+y}{2}\right) \leq \frac{f(x)+f(y)}{2}
$$

for all $x, y \in(a, b)$. Prove that $f$ is convex.

\textbf{Answer}

\clearpage
	
	\section{Problem 25}
	
	If $A \subset R^{k}$ and $B \subset R^{k}$, define $A+B$ to be the set of all sums $\mathbf{x}+\mathbf{y}$ with $\mathbf{x} \in A$, $\mathbf{y} \in B$.
	
	(a) If $K$ is compact and $C$ is closed in $R^{k}$, prove that $K+C$ is closed.
	
	\emph{Hint}: Take $\mathbf{z} \notin K+C$, put $F=\mathbf{z}-C$, the set of all $\mathbf{z}-\mathbf{y}$ with $\mathbf{y} \in C$. Then $K$ and $F$ are disjoint. Choose $\delta$ as in Exercise 21. Show that the open ball with center $\mathbf{z}$ and radius $\delta$ does not intersect $K+C$.
	
	(b) Let $\alpha$ be an irrational real number. Let $C_{1}$ be the set of all integers, let $C_{2}$ be the set of all $n \alpha$ with $n \in C_{1}$. Show that $C_{1}$ and $C_{2}$ are closed subsets of $R^{1}$ whose sum $C_{1}+C_{2}$ is not closed, by showing that $C_{1}+C_{2}$ is a countable dense subset of $R^{1}$.
	
	\textbf{Answer}
	
	\clearpage
	
	
	\section{Problem 26}
	
	Suppose $X, Y, Z$ are metric spaces, and $Y$ is compact. Let $f$ map $X$ into $Y$, let $g$ be a continuous one-to-one mapping of $Y$ into $Z$, and put $h(x)=g(f(x))$ for $x \in X$.
	
	Prove that $f$ is uniformly continuous if $h$ is uniformly continuous.
	
	\emph{Hint}: $g^{-1}$ has compact domain $g(Y)$, and $f(x)=g^{-1}(h(x))$. Prove also that $f$ is continuous if $h$ is continuous. Show (by modifying Example 4.21, or by finding a different example) that the compactness of $Y$ cannot be omitted from the hypotheses, even when $X$ and $Z$ are compact.
	
	\textbf{Answer}
	
	\clearpage
	
\end{document}


