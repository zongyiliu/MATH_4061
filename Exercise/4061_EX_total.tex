\documentclass[10pt]{article}
\usepackage[utf8]{inputenc}
\usepackage[T1]{fontenc}
\usepackage{amsmath}
\usepackage{amsfonts}
\usepackage{amssymb}
\usepackage[version=4]{mhchem}
\usepackage{stmaryrd}
\usepackage{mathrsfs}

\begin{document}
\begin{enumerate}
  \item Suppose $f$ is a real function defined on $R^{1}$ which satisfies
\end{enumerate}

$$
\lim _{h \rightarrow 0}[f(x+h)-f(x-h)]=0
$$

for every $x \in R^{1}$. Does this imply that $f$ is continuous?\\
2. If $f$ is a continuous mapping of a metric space $X$ into a metric space $Y$, prove that

$$
f(\bar{E}) \subset \overline{f(E)}
$$

for every set $E \subset X$. ($\bar{E}$ denotes the closure of $E$.) Show, by an example, that $f(\bar{E})$ can be a proper subset of $\overline{f(E)}$.\\
3. Let $f$ be a continuous real function on a metric space $X$. Let $Z(f)$ (the zero set of $f$) be the set of all $p \in X$ at which $f(p)=0$. Prove that $Z(f)$ is closed.\\
4. Let $f$ and $g$ be continuous mappings of a metric space $X$ into a metric space $Y$,\\
and let $E$ be a dense subset of $X$. Prove that $f(E)$ is dense in $f(X)$. If $g(p)=f(p)$ for all $p \in E$, prove that $g(p)=f(p)$ for all $p \in X$. (In other words, a continuous mapping is determined by its values on a dense subset of its domain.)\\
5. If $f$ is a real continuous function defined on a closed set $E \subset R^{1}$, prove that there exist continuous real functions $g$ on $R^{1}$ such that $g(x)=f(x)$ for all $x \in E$. (Such functions $g$ are called continuous extensions of $f$ from $E$ to $R^{1}$.) Show that the result becomes false if the word "closed" is omitted. Extend the result to vectorvalued functions. Hint: Let the graph of $g$ be a straight line on each of the segments which constitute the complement of $E$ (compare Exercise 29, Chap. 2). The result remains true if $R^{1}$ is replaced by any metric space, but the proof is not so simple.\\
6. If $f$ is defined on $E$, the graph of $f$ is the set of points $(x, f(x))$, for $x \in E$. In particular, if $E$ is a set of real numbers, and $f$ is real-valued, the graph of $f$ is a subset of the plane.

Suppose $E$ is compact, and prove that $f$ is continuous on $E$ if and only if its graph is compact.\\
7. If $E \subset X$ and if $f$ is a function defined on $X$, the restriction of $f$ to $E$ is the function $g$ whose domain of definition is $E$, such that $g(p)=f(p)$ for $p \in E$. Define $f$ and $g$ on $R^{2}$ by: $f(0,0)=g(0,0)=0, f(x, y)=x y^{2} /\left(x^{2}+y^{4}\right), g(x, y)=x y^{2} /\left(x^{2}+y^{6}\right)$ if $(x, y) \neq(0,0)$. Prove that $f$ is bounded on $R^{2}$, that $g$ is unbounded in every neighborhood of $(0,0)$, and that $f$ is not continuous at $(0,0)$; nevertheless, the restrictions of both $f$ and $g$ to every straight line in $R^{2}$ are continuous!\\
8. Let $f$ be a real uniformly continuous function on the bounded set $E$ in $R^{1}$. Prove that $f$ is bounded on $E$.

Show that the conclusion is false if boundedness of $E$ is omitted from the hypothesis.\\
9. Show that the requirement in the definition of uniform continuity can be rephrased as follows, in terms of diameters of sets: To every $\varepsilon>0$ there exists a $\delta>0$ such that diam $f(E)<\varepsilon$ for all $E \subset X$ with diam $E<\delta$.\\
10. Complete the details of the following alternative proof of Theorem 4.19: If $f$ is not uniformly continuous, then for some $\varepsilon>0$ there are sequences $\left\{p_{n}\right\},\left\{q_{n}\right\}$ in $X$ such that $d_{X}\left(p_{n}, q_{n}\right) \rightarrow 0$ but $d_{Y}\left(f\left(p_{n}\right), f\left(q_{n}\right)\right)>\varepsilon$. Use Theorem 2.37 to obtain a contradiction.\\
11. Suppose $f$ is a uniformly continuous mapping of a metric space $X$ into a metric space $Y$ and prove that $\left\{f\left(x_{n}\right)\right\}$ is a Cauchy sequence in $Y$ for every Cauchy sequence $\left\{x_{n}\right\}$ in $X$. Use this result to give an alternative proof of the theorem stated in Exercise 13.\\
12. A uniformly continuous function of a uniformly continuous function is uniformly continuous.

State this more precisely and prove it.\\
13. Let $E$ be a dense subset of a metric space $X$, and let $f$ be a uniformly continuous real function defined on $E$. Prove that $f$ has a continuous extension from $E$ to $X$\\
(see Exercise 5 for terminology). (Uniqueness follows from Exercise 4.) Hint: For each $p \in X$ and each positive integer $n$, let $V_{n}(p)$ be the set of all $q \in E$ with $d(p, q)<1 / n$. Use Exercise 9 to show that the intersection of the closures of the sets $f\left(V_{1}(p)\right), f\left(V_{2}(p)\right), \ldots$, consists of a single point, say $g(p)$, of $R^{1}$. Prove that the function $g$ so defined on $X$ is the desired extension of $f$.

Could the range space $R^{1}$ be replaced by $R^{k}$ ? By any compact metric space? By any complete metric space? By any metric space?\\
14. Let $I=[0,1]$ be the closed unit interval. Suppose $f$ is a continuous mapping of $I$ into $I$. Prove that $f(x)=x$ for at least one $x \in I$.\\
15. Call a mapping of $X$ into $Y$ open if $f(V)$ is an open set in $Y$ whenever $V$ is an open set in $X$.

Prove that every continuous open mapping of $R^{1}$ into $R^{1}$ is monotonic.\\
16. Let [ $x$ ] denote the largest integer contained in $x$, that is, $[x]$ is the integer such that $x-1<[x] \leq x$; and let $(x)=x-[x]$ denote the fractional part of $x$. What discontinuities do the functions $[x]$ and $(x)$ have?\\
17. Let $f$ be a real function defined on $(a, b)$. Prove that the set of points at which $f$ has a simple discontinuity is at most countable. Hint: Let $E$ be the set on which $f(x-)<f(x+)$. With each point $x$ of $E$, associate a triple $(p, q, r)$ of rational numbers such that\\
(a) $f(x-)<p<f(x+)$,\\
(b) $a<q<t<x$ implies $f(t)<p$,\\
(c) $x<t<r<b$ implies $f(t)>p$.

The set of all such triples is countable. Show that each triple is associated with at most one point of $E$. Deal similarly with the other possible types of simple discontinuities.\\
18. Every rational $x$ can be written in the form $x=m / n$, where $n>0$, and $m$ and $n$ are integers without any common divisors. When $x=0$, we take $n=1$. Consider the function $f$ defined on $R^{1}$ by

Prove that $f$ is continuous at every irrational point, and that $f$ has a simple discontinuity at every rational point.\\
19. Suppose $f$ is a real function with domain $R^{1}$ which has the intermediate value property: If $f(a)<c<f(b)$, then $f(x)=c$ for some $x$ between $a$ and $b$.

Suppose also, for every rational $r$, that the set of all $x$ with $f(x)=r$ is closed.\\
Prove that $f$ is continuous.\\
Hint: If $x_{n} \rightarrow x_{0}$ but $f\left(x_{n}\right)>r>f\left(x_{0}\right)$ for some $r$ and all $n$, then $f\left(t_{n}\right)=r$ for some $t_{n}$ between $x_{0}$ and $x_{n}$; thus $t_{n} \rightarrow x_{0}$. Find a contradiction. (N. J. Fine, Amer. Math. Monthly, vol. 73, 1966, p. 782.)\\
20. If $E$ is a nonempty subset of a metric space $X$, define the distance from $x \in X$ to $E$ by

$$
\rho_{E}(x)=\inf _{z \in E} d(x, z) .
$$

(a) Prove that $\rho_{E}(x)=0$ if and only if $x \in \bar{E}$.\\
(b) Prove that $\rho_{E}$ is a uniformly continuous function on $X$, by showing that

$$
\left|\rho_{E}(x)-\rho_{E}(y)\right| \leq d(x, y)
$$

for all $x \in X, y \in X$.\\
Hint: $\rho_{E}(x) \leq d(x, z) \leq d(x, y)+d(y, z)$, so that

$$
\rho_{E}(x) \leq d(x, y)+\rho_{E}(y)
$$

\begin{enumerate}
  \setcounter{enumi}{20}
  \item Suppose $K$ and $F$ are disjoint sets in a metric space $X, K$ is compact, $F$ is closed. Prove that there exists $\delta>0$ such that $d(p, q)>\delta$ if $p \in K, q \in F$. Hint: $\rho_{F}$ is a continuous positive function on $K$.
\end{enumerate}

Show that the conclusion may fail for two disjoint closed sets if neither is compact.\\
22. Let $A$ and $B$ be disjoint nonempty closed sets in a metric space $X$, and define

$$
f(p)=\frac{\rho_{A}(p)}{\rho_{A}(p)+\rho_{B}(p)} \quad(p \in X)
$$

Show that $f$ is a continuous function on $X$ whose range lies in $[0,1]$, that $f(p)=0$ precisely on $A$ and $f(p)=1$ precisely on $B$. This establishes a converse of Exercise 3: Every closed set $A \subset X$ is $Z(f)$ for some continuous real $f$ on $X$. Setting

$$
V=f^{-1}\left(\left[0, \frac{1}{2}\right)\right), \quad W=f^{-1}\left(\left(\frac{1}{2}, 1\right]\right),
$$

show that $V$ and $W$ are open and disjoint, and that $A \subset V, B \subset W$. (Thus pairs of disjoint closed sets in a metric space can be covered by pairs of disjoint open sets. This property of metric spaces is called normality.)\\
23. A real-valued function $f$ defined in $(a, b)$ is said to be convex if

$$
f(\lambda x+(1-\lambda) y) \leq \lambda f(x)+(1-\lambda) f(y)
$$

whenever $a<x<b, a<y<b, 0<\lambda<1$. Prove that every convex function is continuous. Prove that every increasing convex function of a convex function is convex. (For example, if $f$ is convex, so is $e^{f}$.)

If $f$ is convex in ($a, b$) and if $a<s<t<u<b$, show that

$$
\frac{f(t)-f(s)}{t-s} \leq \frac{f(u)-f(s)}{u-s} \leq \frac{f(u)-f(t)}{u-t}
$$

\begin{enumerate}
  \setcounter{enumi}{23}
  \item Assume that $f$ is a continuous real function defined in ($a, b$) such that
\end{enumerate}

$$
f\left(\frac{x+y}{2}\right) \leq \frac{f(x)+f(y)}{2}
$$

for all $x, y \in(a, b)$. Prove that $f$ is convex.\\
25. If $A \subset R^{k}$ and $B \subset R^{k}$, define $A+B$ to be the set of all sums $\mathbf{x}+\mathbf{y}$ with $\mathbf{x} \in A$, $\mathbf{y} \in B$.\\
(a) If $K$ is compact and $C$ is closed in $R^{k}$, prove that $K+C$ is closed.

Hint: Take $\mathbf{z} \notin K+C$, put $F=\mathbf{z}-C$, the set of all $\mathbf{z}-\mathbf{y}$ with $\mathbf{y} \in C$. Then $K$ and $F$ are disjoint. Choose $\delta$ as in Exercise 21. Show that the open ball with center $\mathbf{z}$ and radius $\delta$ does not intersect $K+C$.\\
(b) Let $\alpha$ be an irrational real number. Let $C_{1}$ be the set of all integers, let $C_{2}$ be the set of all $n \alpha$ with $n \in C_{1}$. Show that $C_{1}$ and $C_{2}$ are closed subsets of $R^{1}$ whose sum $C_{1}+C_{2}$ is not closed, by showing that $C_{1}+C_{2}$ is a countable dense subset of $R^{1}$.\\
26. Suppose $X, Y, Z$ are metric spaces, and $Y$ is compact. Let $f$ map $X$ into $Y$, let $g$ be a continuous one-to-one mapping of $Y$ into $Z$, and put $h(x)=g(f(x))$ for $x \in X$.

Prove that $f$ is uniformly continuous if $h$ is uniformly continuous.\\
Hint: $g^{-1}$ has compact domain $g(Y)$, and $f(x)=g^{-1}(h(x))$.\\
Prove also that $f$ is continuous if $h$ is continuous.\\
Show (by modifying Example 4.21, or by finding a different example) that the compactness of $Y$ cannot be omitted from the hypotheses, even when $X$ and $Z$ are compact.

\begin{enumerate}
  \item Let $f$ be defined for all real $x$, and suppose that
\end{enumerate}

$$
|f(x)-f(y)| \leq(x-y)^{2}
$$

for all real $x$ and $y$. Prove that $f$ is constant.\\
2. Suppose $f^{\prime}(x)>0$ in $(a, b)$. Prove that $f$ is strictly increasing in $(a, b)$, and let $g$ be its inverse function. Prove that $g$ is differentiable, and that

$$
g^{\prime}(f(x))=\frac{1}{f^{\prime}(x)} \quad(a<x<b)
$$

\begin{enumerate}
  \setcounter{enumi}{2}
  \item Suppose $g$ is a real function on $R^{1}$, with bounded derivative (say $\left|g^{\prime}\right| \leq M$). Fix $\varepsilon>0$, and define $f(x)=x+\varepsilon g(x)$. Prove that $f$ is one-to-one if $\varepsilon$ is small enough.\\
(A set of admissible values of $\varepsilon$ can be determined which depends only on $M$.)
  \item If
\end{enumerate}

$$
C_{0}+\frac{C_{1}}{2}+\cdots+\frac{C_{n-1}}{n}+\frac{C_{n}}{n+1}=0
$$

where $C_{0}, \ldots, C_{n}$ are real constants, prove that the equation

$$
C_{0}+C_{1} x+\cdots+C_{n-1} x^{n-1}+C_{n} x^{n}=0
$$

has at least one real root between 0 and 1 .\\
5. Suppose $f$ is defined and differentiable for every $x>0$, and $f^{\prime}(x) \rightarrow 0$ as $x \rightarrow+\infty$. Put $g(x)=f(x+1)-f(x)$. Prove that $g(x) \rightarrow 0$ as $x \rightarrow+\infty$.\\
6. Suppose\\
(a) $f$ is continuous for $x \geq 0$,\\
(b) $f^{\prime}(x)$ exists for $x>0$,\\
(c) $f(0)=0$,\\
(d) $f^{\prime}$ is monotonically increasing.

Put

$$
g(x)=\frac{f(x)}{x} \quad(x>0)
$$

and prove that $g$ is monotonically increasing.\\
7. Suppose $f^{\prime}(x), g^{\prime}(x)$ exist, $g^{\prime}(x) \neq 0$, and $f(x)=g(x)=0$. Prove that

$$
\lim _{t \rightarrow x} \frac{f(t)}{g(t)}=\frac{f^{\prime}(x)}{g^{\prime}(x)}
$$

(This holds also for complex functions.)\\
8. Suppose $f^{\prime}$ is continuous on $[a, b]$ and $\varepsilon>0$. Prove that there exists $\delta>0$ such that

$$
\left|\frac{f(t)-f(x)}{t-x}-f^{\prime}(x)\right|<\varepsilon
$$

whenever $0<|t-x|<\delta, a \leq x \leq b, a \leq t \leq b$. (This could be expressed by saying that $f$ is uniformly differentiable on $[a, b]$ if $f^{\prime}$ is continuous on $[a, b]$.) Does this hold for vector-valued functions too?\\
9. Let $f$ be a continuous real function on $R^{1}$, of which it is known that $f^{\prime}(x)$ exists for all $x \neq 0$ and that $f^{\prime}(x) \rightarrow 3$ as $x \rightarrow 0$. Does it follow that $f^{\prime}(0)$ exists?\\
10. Suppose $f$ and $g$ are complex differentiable functions on $(0,1), f(x) \rightarrow 0, g(x) \rightarrow 0$, $f^{\prime}(x) \rightarrow A, g^{\prime}(x) \rightarrow B$ as $x \rightarrow 0$, where $A$ and $B$ are complex numbers, $B \neq 0$. Prove that

$$
\lim _{x \rightarrow 0} \frac{f(x)}{g(x)}=\frac{A}{B}
$$

Compare with Example 5.18. Hint:

$$
\frac{f(x)}{g(x)}=\left\{\frac{f(x)}{x}-A\right\} \cdot \frac{x}{g(x)}+A \cdot \frac{x}{g(x)} .
$$

Apply Theorem 5.13 to the real and imaginary parts of $f(x) / x$ and $g(x) / x$.\\
11. Suppose $f$ is defined in a neighborhood of $x$, and suppose $f^{\prime \prime}(x)$ exists. Show that

$$
\lim _{h \rightarrow 0} \frac{f(x+h)+f(x-h)-2 f(x)}{h^{2}}=f^{\prime \prime}(x)
$$

Show by an example that the limit may exist even if $f^{\prime \prime}(x)$ does not.\\
Hint: Use Theorem 5.13.\\
12. If $f(x)=|x|^{3}$, compute $f^{\prime}(x), f^{\prime \prime}(x)$ for all real $x$, and show that $f^{(3)}(0)$ does not exist.\\
13. Suppose $a$ and $c$ are real numbers, $c>0$, and $f$ is defined on $[-1,1]$ by

$$
f(x)= \begin{cases}x^{a} \sin \left(|x|^{-c}\right) & (\text { if } x \neq 0) \\ 0 & (\text { if } x=0)\end{cases}
$$

Prove the following statements:\\
(a) $\boldsymbol{f}$ is continuous if and only if $\boldsymbol{a}>\mathbf{0}$.\\
(b) $f^{\prime}(0)$ exists if and only if $a>1$.\\
(c) $f^{\prime}$ is bounded if and only if $a \geq 1+c$.\\
(d) $f^{\prime}$ is continuous if and only if $a>1+c$.\\
(e) $f^{\prime \prime}(0)$ exists if and only if $a>2+c$.\\
(f) $f^{\prime \prime}$ is bounded if and only if $a \geq 2+2 c$.\\
(g) $f^{\prime \prime}$ is continuous if and only if $a>2+2 c$.\\
14. Let $f$ be a differentiable real function defined in ($a, b$). Prove that $f$ is convex if and only if $f^{\prime}$ is monotonically increasing. Assume next that $f^{\prime \prime}(x)$ exists for every $x \in(a, b)$, and prove that $f$ is convex if and only if $f^{\prime \prime}(x) \geq 0$ for all $x \in(a, b)$.\\
15. Suppose $a \in R^{1}, f$ is a twice-differentiable real function on ($a, \infty$), and $M_{0}, M_{1}, M_{2}$ are the least upper bounds of $|f(x)|,\left|f^{\prime}(x)\right|,\left|f^{\prime \prime}(x)\right|$, respectively, on $(a, \infty)$. Prove that

$$
M_{1}^{2} \leq 4 M_{0} M_{2}
$$

Hint: If $h>0$, Taylor's theorem shows that

$$
f^{\prime}(x)=\frac{1}{2 h}[f(x+2 h)-f(x)]-h f^{\prime \prime}(\xi)
$$

for some $\xi \in(x, x+2 h)$. Hence

$$
\left|f^{\prime}(x)\right| \leq h M_{2}+\frac{M_{0}}{h}
$$

To show that $M_{1}^{2}=4 M_{0} M_{2}$ can actually happen, take $a=-1$, define

$$
f(x)= \begin{cases}2 x^{2}-1 & (-1<x<0) \\ \frac{x^{2}-1}{x^{2}+1} & (0 \leq x<\infty)\end{cases}
$$

and show that $M_{0}=1, M_{1}=4, M_{2}=4$.\\
Does $M_{1}^{2} \leq 4 M_{0} M_{2}$ hold for vector-valued function too?\\
16. Suppose $f$ is twice-differentiable on $(0, \infty), f^{\prime \prime}$ is bounded on $(0, \infty)$, and $f(x) \rightarrow 0$ as $x \rightarrow \infty$. Prove that $f^{\prime}(x) \rightarrow 0$ as $x \rightarrow \infty$.

Hint: Let $a \rightarrow \infty$ in Exercise 15.\\
17. Suppose $f$ is a real, three times differentiable function on $[-1,1]$, such that

$$
f(-1)=0, \quad f(0)=0, \quad f(1)=1, \quad f^{\prime}(0)=0
$$

Prove that $f^{(3)}(x) \geq 3$ for some $x \in(-1,1)$.\\
Note that equality holds for $\frac{1}{2}\left(x^{3}+x^{2}\right)$.\\
Hint: Use Theorem 5.15, with $\alpha=0$ and $\beta= \pm 1$, to show that there exist $s \in(0,1)$ and $t \in(-1,0)$ such that

$$
f^{(3)}(s)+f^{(3)}(t)=6
$$

\begin{enumerate}
  \setcounter{enumi}{17}
  \item Suppose $f$ is a real function on $[a, b], n$ is a positive integer, and $f^{(n-1)}$ exists for every $t \in[a, b]$. Let $\alpha, \beta$, and $P$ be as in Taylor's theorem (5.15). Define
\end{enumerate}

$$
Q(t)=\frac{f(t)-f(\beta)}{t-\beta}
$$

for $t \in[a, b], t \neq \beta$, differentiate

$$
f(t)-f(\beta)=(t-\beta) Q(t)
$$

$n-1$ times at $t=\alpha$, and derive the following version of Taylor's theorem:

$$
f(\beta)=P(\beta)+\frac{Q^{(n-1)}(\alpha)}{(n-1)!}(\beta-\alpha)^{n}
$$

\begin{enumerate}
  \setcounter{enumi}{18}
  \item Suppose $f$ is defined in $(-1,1)$ and $f^{\prime}(0)$ exists. Suppose $-1<\alpha_{n}<\beta_{n}<1$, $\alpha_{n} \rightarrow 0$, and $\beta_{n} \rightarrow 0$ as $n \rightarrow \infty$. Define the difference quotients
\end{enumerate}

$$
D_{n}=\frac{f\left(\beta_{n}\right)-f\left(\alpha_{n}\right)}{\beta_{n}-\alpha_{n}}
$$

Prove the following statements:\\
(a) If $\alpha_{n}<0<\beta_{n}$, then $\lim D_{n}=f^{\prime}(0)$.\\
(b) If $0<\alpha_{n}<\beta_{n}$ and $\left\{\beta_{n} /\left(\beta_{n}-\alpha_{n}\right)\right\}$ is bounded, then $\lim D_{n}=f^{\prime}(0)$.\\
(c) If $f^{\prime}$ is continuous in $(-1,1)$, then $\lim D_{n}=f^{\prime}(0)$.

Give an example in which $f$ is differentiable in $(-1,1)$ (but $f^{\prime}$ is not continuous at 0 ) and in which $\alpha_{n}, \beta_{n}$ tend to 0 in such a way that $\lim D_{n}$ exists but is different from $f^{\prime}(0)$.\\
20. Formulate and prove an inequality which follows from Taylor's theorem and which remains valid for vector-valued functions.\\
21. Let $E$ be a closed subset of $R^{1}$. We saw in Exercise 22, Chap. 4, that there is a real continuous function $f$ on $R^{1}$ whose zero set is $E$. Is it possible, for each closed set $E$, to find such an $f$ which is differentiable on $R^{1}$, or one which is $n$ times differentiable, or even one which has derivatives of all orders on $R^{1}$ ?\\
22. Suppose $f$ is a real function on $(-\infty, \infty)$. Call $x$ a fixed point of $f$ if $f(x)=x$. (a) If $f$ is differentiable and $f^{\prime}(t) \neq 1$ for every real $t$, prove that $f$ has at most one fixed point.\\
(b) Show that the function $f$ defined by

$$
f(t)=t+\left(1+e^{t}\right)^{-1}
$$

has no fixed point, although $0<f^{\prime}(t)<1$ for all real $t$.\\
(c) However, if there is a constant $A<1$ such that $\left|f^{\prime}(t)\right| \leq A$ for all real $t$, prove that a fixed point $x$ of $f$ exists, and that $x=\lim x_{n}$, where $x_{1}$ is an arbitrary real number and

$$
x_{n+1}=f\left(x_{n}\right)
$$

for $n=1,2,3, \ldots$.\\
(d) Show that the process described in (c) can be visualized by the zig-zag path

$$
\left(x_{1}, x_{2}\right) \rightarrow\left(x_{2}, x_{2}\right) \rightarrow\left(x_{2}, x_{3}\right) \rightarrow\left(x_{3}, x_{3}\right) \rightarrow\left(x_{3}, x_{4}\right) \rightarrow \cdots .
$$

\begin{enumerate}
  \setcounter{enumi}{22}
  \item The function $f$ defined by
\end{enumerate}

$$
f(x)=\frac{x^{3}+1}{3}
$$

has three fixed points, say $\alpha, \beta, \gamma$, where

$$
-2<\alpha<-1, \quad 0<\beta<1, \quad 1<\gamma<2 .
$$

For arbitrarily chosen $x_{1}$, define $\left\{x_{n}\right\}$ by setting $x_{n+1}=f\left(x_{n}\right)$.\\
(a) If $x_{1}<\alpha$, prove that $x_{n} \rightarrow-\infty$ as $n \rightarrow \infty$.\\
(b) If $\alpha<x_{1}<\gamma$, prove that $x_{n} \rightarrow \beta$ as $n \rightarrow \infty$.\\
(c) If $\gamma<x_{1}$, prove that $x_{n} \rightarrow+\infty$ as $n \rightarrow \infty$.

Thus $\beta$ can be located by this method, but $\alpha$ and $\gamma$ cannot.\\
24. The process described in part (c) of Exercise 22 can of course also be applied to functions that map $(0, \infty)$ to $(0, \infty)$.

Fix some $\alpha>1$, and put

$$
f(x)=\frac{1}{2}\left(x+\frac{\alpha}{x}\right), \quad g(x)=\frac{\alpha+x}{1+x} .
$$

Both $f$ and $g$ have $\sqrt{\alpha}$ as their only fixed point in $(0, \infty)$. Try to explain, on the basis of properties of $f$ and $g$, why the convergence in Exercise 16, Chap. 3, is so much more rapid than it is in Exercise 17. (Compare $f^{\prime}$ and $g^{\prime}$, draw the zig-zags suggested in Exercise 22.)

Do the same when $0<\alpha<1$.\\
25. Suppose $f$ is twice differentiable on $[a, b], f(a)<0, f(b)>0, f^{\prime}(x) \geq \delta>0$, and $0 \leq f^{\prime \prime}(x) \leq M$ for all $x \in[a, b]$. Let $\xi$ be the unique point in $(a, b)$ at which $f(\xi)=0$.

Complete the details in the following outline of Newton's method for computing $\xi$.\\
(a) Choose $x_{1} \in(\xi, b)$, and define $\left\{x_{n}\right\}$ by

$$
x_{n+1}=x_{n}-\frac{f\left(x_{n}\right)}{f^{\prime}\left(x_{n}\right)}
$$

Interpret this geometrically, in terms of a tangent to the graph of $f$.\\
(b) Prove that $x_{n+1}<x_{n}$ and that

$$
\lim _{n \rightarrow \infty} x_{n}=\xi
$$

(c) Use Taylor's theorem to show that

$$
x_{n+1}-\xi=\frac{f^{\prime \prime}\left(t_{n}\right)}{2 f^{\prime}\left(x_{n}\right)}\left(x_{n}-\xi\right)^{2}
$$

for some $t_{n} \in\left(\xi, x_{n}\right)$.\\
(d) If $A=M / 2 \delta$, deduce that

$$
0 \leq x_{n+1}-\xi \leq \frac{1}{A}\left[A\left(x_{1}-\xi\right)\right]^{2 n}
$$

(Compare with Exercises 16 and 18, Chap. 3.)\\
(e) Show that Newton's method amounts to finding a fixed point of the function $g$ defined by

$$
g(x)=x-\frac{f(x)}{f^{\prime}(x)}
$$

How does $g^{\prime}(x)$ behave for $x$ near $\xi$ ?\\
(f) Put $f(x)=x^{1 / 3}$ on $(-\infty, \infty)$ and try Newton's method. What happens?\\
26. Suppose $f$ is differentiable on $[a, b], f(a)=0$, and there is a real number $A$ such that $\left|f^{\prime}(x)\right| \leq A|f(x)|$ on $[a, b]$. Prove that $f(x)=0$ for all $x \in[a, b]$. Hint: Fix $x_{0} \in[a, b]$, let

$$
M_{0}=\sup |f(x)|, \quad M_{1}=\sup \left|f^{\prime}(x)\right|
$$

for $a \leq x \leq x_{0}$. For any such $x$,

$$
|f(x)| \leq M_{1}\left(x_{0}-a\right) \leq A\left(x_{0}-a\right) M_{0} .
$$

Hence $M_{0}=0$ if $A\left(x_{0}-a\right)<1$. That is, $f=0$ on $\left[a, x_{0}\right]$. Proceed.\\
27. Let $\phi$ be a real function defined on a rectangle $R$ in the plane, given by $a \leq x \leq b$, $\alpha \leq y \leq \beta$. A solution of the initial-value problem

$$
y^{\prime}=\phi(x, y), \quad y(a)=c \quad(\alpha \leq c \leq \beta)
$$

is, by definition, a differentiable function $f$ on $[a, b]$ such that $f(a)=c, \alpha \leq f(x) \leq \beta$, and

$$
f^{\prime}(x)=\phi(x, f(x)) \quad(a \leq x \leq b) .
$$

Prove that such a problem has at most one solution if there is a constant $A$ such that

$$
\left|\phi\left(x, y_{2}\right)-\phi\left(x, y_{1}\right)\right| \leq A\left|y_{2}-y_{1}\right|
$$

whenever $\left(x, y_{1}\right) \in R$ and $\left(x, y_{2}\right) \in R$.\\
Hint: Apply Exercise 26 to the difference of two solutions. Note that this uniqueness theorem does not hold for the initial-value problem

$$
y^{\prime}=y^{1 / 2}, \quad y(0)=0,
$$

which has two solutions: $f(x)=0$ and $f(x)=x^{2} / 4$. Find all other solutions.\\
28. Formulate and prove an analogous uniqueness theorem for systems of differential equations of the form

$$
y_{j}^{\prime}=\phi_{j}\left(x, y_{1}, \ldots, y_{k}\right), \quad y_{j}(a)=c_{j} \quad(j=1, \ldots, k)
$$

Note that this can be rewritten in the form

$$
\mathbf{y}^{\prime}=\boldsymbol{\phi}(x, \mathbf{y}), \quad \mathbf{y}(a)=\mathbf{c}
$$

where $\mathbf{y}=\left(y_{1}, \ldots, y_{k}\right)$ ranges over a $k$-cell, $\boldsymbol{\phi}$ is the mapping of a $(k+1)$-cell into the Euclidean $k$-space whose components are the functions $\phi_{1}, \ldots, \phi_{k}$, and $\mathbf{c}$ is the vector $\left(c_{1}, \ldots, c_{k}\right)$. Use Exercise 26, for vector-valued functions.\\
29. Specialize Exercise 28 by considering the system

$$
\begin{aligned}
y_{j}^{\prime} & =y_{j+1} \quad(j=1, \ldots, k-1) \\
y_{k}^{\prime} & =f(x)-\sum_{j=1}^{k} g_{j}(x) y_{j}
\end{aligned}
$$

where $f, g_{1}, \ldots, g_{k}$ are continuous real functions on $[a, b]$, and derive a uniqueness theorem for solutions of the equation

$$
y^{(k)}+g_{k}(x) y^{(k-1)}+\cdots+g_{2}(x) y^{\prime}+g_{1}(x) y=f(x)
$$

subject to initial conditions

$$
y(a)=c_{1}, \quad y^{\prime}(a)=c_{2}, \quad \ldots, \quad y^{(k-1)}(a)=c_{k}
$$

(a) If $f \in \mathscr{R}$ on $[0,1]$, show that this definition of the integral agrees with the old one.\\
(b) Construct a function $f$ such that the above limit exists, although it fails to exist with $|f|$ in place of $f$.\\
8. Suppose $f \in \mathscr{R}$ on $[a, b]$ for every $b>a$ where $a$ is fixed. Define

$$
\int_{a}^{\infty} f(x) d x=\lim _{b \rightarrow \infty} \int_{a}^{b} f(x) d x
$$

if this limit exists (and is finite). In that case, we say that the integral on the left converges. If it also converges after $f$ has been replaced by $|f|$, it is said to converge absolutely.

Assume that $f(x) \geq 0$ and that $f$ decreases monotonically on $[1, \infty)$. Prove that

$$
\int_{1}^{\infty} f(x) d x
$$

converges if and only if

$$
\sum_{n=1}^{\infty} f(n)
$$

converges. (This is the so-called "integral test" for convergence of series.)\\
9. Show that integration by parts can sometimes be applied to the "improper" integrals defined in Exercises 7 and 8. (State appropriate hypotheses, formulate a theorem, and prove it.) For instance show that

$$
\int_{0}^{\infty} \frac{\cos x}{1+x} d x=\int_{0}^{\infty} \frac{\sin x}{(1+x)^{2}} d x
$$

Show that one of these integrals converges absolutely, but that the other does not.\\
10. Let $p$ and $q$ be positive real numbers such that

$$
\frac{1}{p}+\frac{1}{q}=1 .
$$

Prove the following statements.\\
(a) If $u \geq 0$ and $v \geq 0$, then

$$
u v \leq \frac{u^{p}}{p}+\frac{v^{q}}{q}
$$

Equality holds if and only if $u^{p}=v^{q}$.\\
(b) If $f \in \mathscr{R}(\alpha), g \in \mathscr{R}(\alpha), f \geq 0, g \geq 0$, and

$$
\int_{a}^{b} f^{p} d \alpha=1=\int_{a}^{b} g^{q} d \alpha
$$

then

$$
\int_{a}^{b} f g d \alpha \leq 1
$$

(c) If $f$ and $g$ are complex functions in $\mathscr{R}(\alpha)$, then

$$
\left|\int_{a}^{b} f g d \alpha\right| \leq\left\{\int_{a}^{b}|f|^{p} d \alpha\right\}^{1 / p}\left\{\int_{a}^{b}|g|^{a} d \alpha\right\}^{1 / a}
$$

This is Hölder's inequality. When $p=q=2$ it is usually called the Schwarz inequality. (Note that Theorem 1.35 is a very special case of this.)\\
(d) Show that Hölder's inequality is also true for the "improper" integrals described in Exercises 7 and 8.\\
11. Let $\alpha$ be a fixed increasing function on $[a, b]$. For $u \in \mathscr{R}(\alpha)$, define

$$
\|u\|_{2}=\left\{\int_{a}^{b}|u|^{2} d \alpha\right\}^{1 / 2}
$$

Suppose $f, g, h \in \mathscr{R}(\alpha)$, and prove the triangle inequality

$$
\|f-h\|_{2} \leq\|f-g\|_{2}+\|g-h\|_{2}
$$

as a consequence of the Schwarz inequality, as in the proof of Theorem 1.37.\\
12. With the notations of Exercise 11, suppose $f \in \mathscr{R}(\alpha)$ and $\varepsilon>0$. Prove that there exists a continuous function $g$ on $[a, b]$ such that $\|f-g\|_{2}<\varepsilon$.

Hint: Let $P=\left\{x_{0}, \ldots, x_{n}\right\}$ be a suitable partition of $[a, b]$, define

$$
g(t)=\frac{x_{i}-t}{\Delta x_{i}} f\left(x_{i-1}\right)+\frac{t-x_{i-1}}{\Delta x_{i}} f\left(x_{i}\right)
$$

if $x_{i-1} \leq t \leq x_{i}$.\\
13. Define

$$
f(x)=\int_{x}^{x+1} \sin \left(t^{2}\right) d t
$$

(a) Prove that $|f(x)|<1 / x$ if $x>0$.

Hint: Put $t^{2}=u$ and integrate by parts, to show that $f(x)$ is equal to

$$
\frac{\cos \left(x^{2}\right)}{2 x}-\frac{\cos \left[(x+1)^{2}\right]}{2(x+1)}-\int_{x^{2}}^{(x+1)^{2}} \frac{\cos u}{4 u^{3 / 2}} d u
$$

Replace $\cos u$ by -1 .\\
(b) Prove that

$$
2 x f(x)=\cos \left(x^{2}\right)-\cos \left[(x+1)^{2}\right]+r(x)
$$

where $|r(x)|<c / x$ and $c$ is a constant.\\
(c) Find the upper and lower limits of $x f(x)$, as $x \rightarrow \infty$.\\
(d) Does $\int_{0}^{\infty} \sin \left(t^{2}\right) d t$ converge?\\
14. Deal similarly with

$$
f(x)=\int_{x}^{x+1} \sin \left(e^{t}\right) d t
$$

Show that

$$
e^{x}|f(x)|<2
$$

and that

$$
e^{x} f(x)=\cos \left(e^{x}\right)-e^{-1} \cos \left(e^{x+1}\right)+r(x)
$$

where $|r(x)|<C e^{-x}$, for some constant $C$.\\
15. Suppose $f$ is a real, continuously differentiable function on $[a, b], f(a)=f(b)=0$, and

$$
\int_{a}^{b} f^{2}(x) d x=1
$$

Prove that

$$
\int_{a}^{b} x f(x) f^{\prime}(x) d x=-\frac{1}{2}
$$

and that

$$
\int_{a}^{b}\left[f^{\prime}(x)\right]^{2} d x \cdot \int_{a}^{b} x^{2} f^{2}(x) d x>4
$$

\begin{enumerate}
  \setcounter{enumi}{15}
  \item For $1<s<\infty$, define
\end{enumerate}

$$
\zeta(s)=\sum_{n=1}^{\infty} \frac{1}{n^{s}} .
$$

(This is Riemann's zeta function, of great importance in the study of the distribution of prime numbers.) Prove that\\
(a) $\zeta(s)=s \int_{1}^{\infty} \frac{[x]}{x^{s+1}} d x$\\
and that\\
(b) $\zeta(s)=\frac{s}{s-1}-s \int_{1}^{\infty} \frac{x-[x]}{x^{s+1}} d x$,\\
where $[x]$ denotes the greatest integer $\leq x$.\\
Prove that the integral in $(b)$ converges for all $s>0$.\\
Hint: To prove $(a)$, compute the difference between the integral over $[1, N]$ and the $N$ th partial sum of the series that defines $\zeta(s)$.\\
17. Suppose $\alpha$ increases monotonically on $[a, b], g$ is continuous, and $g(x)=G^{\prime}(x)$ for $a \leq x \leq b$. Prove that

$$
\int_{a}^{b} \alpha(x) g(x) d x=G(b) \alpha(b)-G(a) \alpha(a)-\int_{a}^{b} G d \alpha
$$

Hint: Take $g$ real, without loss of generality. Given $P=\left\{x_{0}, x_{1}, \ldots, x_{n}\right\}$, choose $t_{i} \in\left(x_{i-1}, x_{i}\right)$ so that $g\left(t_{i}\right) \Delta x_{i}=G\left(x_{i}\right)-G\left(x_{i-1}\right)$. Show that

$$
\sum_{i=1}^{n} \alpha\left(x_{i}\right) g\left(t_{i}\right) \Delta x_{i}=G(b) \alpha(b)-G(a) \alpha(a)-\sum_{i=1}^{n} G\left(x_{i-1}\right) \Delta \alpha_{i} .
$$

\begin{enumerate}
  \setcounter{enumi}{17}
  \item Let $\gamma_{1}, \gamma_{2}, \gamma_{3}$ be curves in the complex plane, defined on $[0,2 \pi]$ by
\end{enumerate}

$$
\gamma_{1}(t)=e^{i t}, \quad \gamma_{2}(t)=e^{2 i t}, \quad \gamma_{3}(t)=e^{2 \pi i t \sin (1 / t)} .
$$

Show that these three curves have the same range, that $\gamma_{1}$ and $\gamma_{2}$ are rectifiable, that the length of $\gamma_{1}$ is $2 \pi$, that the length of $\gamma_{2}$ is $4 \pi$, and that $\gamma_{3}$ is not rectifiable.\\
19. Let $\gamma_{1}$ be a curve in $R^{k}$, defined on $[a, b]$; let $\phi$ be a continuous 1-1 mapping of $[c, d]$ onto $[a, b]$, such that $\phi(c)=a$; and define $\gamma_{2}(s)=\gamma_{1}(\phi(s))$. Prove that $\gamma_{2}$ is an arc, a closed curve, or a rectifiable curve if and only if the same is true of $\gamma_{1}$. Prove that $\gamma_{2}$ and $\gamma_{1}$ have the same length.


\end{document}
