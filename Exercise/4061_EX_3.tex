\documentclass[letterpaper]{article} 
\usepackage[utf8]{inputenc}
\usepackage[T1]{fontenc}
\usepackage{amsmath}
\usepackage{amsfonts}
\usepackage{amssymb}
\usepackage{array}
\usepackage{booktabs}
\usepackage{hyperref}
\usepackage[version=4]{mhchem}
\usepackage{stmaryrd}
\usepackage[dvipsnames]{xcolor}
\colorlet{LightRubineRed}{RubineRed!70}
\colorlet{Mycolor1}{green!10!orange}
\definecolor{Mycolor2}{HTML}{00F9DE}
\usepackage{graphicx}
\usepackage{amsmath}
\usepackage{graphicx}
\usepackage{capt-of}
\usepackage{lipsum}
\usepackage{fancyvrb}
\usepackage{tabularx}
\usepackage{listings}
\usepackage[export]{adjustbox}
\graphicspath{ {./images/} }
\usepackage[utf8]{inputenc}
\usepackage[english]{babel}
\usepackage{float}
\usepackage{lipsum}
\usepackage{graphicx}
\usepackage{float}
\usepackage[margin=0.7in]{geometry}
\usepackage{amsmath}
\usepackage{graphicx}
\usepackage{capt-of}
\usepackage{tcolorbox}
\usepackage{lipsum}
\usepackage{graphicx}
\usepackage{float}
\usepackage{listings}
\usepackage{hyperref} 
\usepackage{xcolor} % For custom colors
\lstset{
	language=Python,                % Choose the language (e.g., Python, C, R)
	basicstyle=\ttfamily\small, % Font size and type
	keywordstyle=\color{blue},  % Keywords color
	commentstyle=\color{gray},  % Comments color
	stringstyle=\color{red},    % String color
	numbers=left,               % Line numbers
	numberstyle=\tiny\color{gray}, % Line number style
	stepnumber=1,               % Numbering step
	breaklines=true,            % Auto line break
	backgroundcolor=\color{black!5}, % Light gray background
	frame=single,               % Frame around the code
}
\usepackage{float}
\usepackage[]{amsthm} %lets us use \begin{proof}
\usepackage[]{amssymb} %gives us the character \varnothing

	\title{Exercise 3, MATH 4061}
	\author{Zongyi Liu}
	\date{Mon, Oct 29, 2025}
	
	\begin{document}
		\maketitle
		\section{Problem 1}
		Prove that convergence of $\{s_n\}$ implies convergence of $\{|s_n|\}$. Is the converse true?
		
		\textbf{Answer}
		
		\clearpage
		
		\section{Problem 2} 
		Calculate $\displaystyle \lim_{n\to\infty}\left(\sqrt{n^{2}+n}-n\right)$.
		
		\textbf{Answer}
		
		\clearpage
		
		\section{Problem 3}
		If $s_1=\sqrt{2}$ and
		\[
		s_{n+1}=\sqrt{2+\sqrt{s_n}}\qquad (n=1,2,3,\dots),
		\]
		prove that $\{s_n\}$ converges, and that $s_n<2$ for $n=1,2,3,\dots$.
		
		
		\textbf{Answer}
		
		\clearpage
		
		\section{Problem 4} 
		Find the upper and lower limits of the sequence $\{s_n\}$ defined by
		\[
		s_1=0,\qquad
		s_{2m}=\frac{s_{2m-1}}{2},\qquad
		s_{2m+1}=\frac12+s_{2m}.
		\]
		
		\textbf{Answer}
		
		\clearpage
		
		\section{Problem 5} 
		For any two real sequences $\{a_n\}$, $\{b_n\}$, prove that
		\[
		\limsup_{n\to\infty}(a_n+b_n)
		\le \limsup_{n\to\infty} a_n + \limsup_{n\to\infty} b_n,
		\]
		provided the sum on the right is not of the form $\infty-\infty$.
		
		
		\textbf{Answer}
		
		\clearpage
		
		\section{Problem 6}
		Investigate the behavior (convergence or divergence) of $\sum a_n$ if
		\begin{enumerate}
			\item[(a)] $a_n=\sqrt{n+1}-\sqrt{n}$;
			\item[(b)] $a_n=\dfrac{\sqrt{n+1}-\sqrt{n}}{n}$;
			\item[(c)] $a_n=\bigl(\sqrt[n]{n}-1\bigr)^n$;
			\item[(d)] $a_n=\dfrac{1}{1+z^{\,n}}$, for complex values of $z$.
		\end{enumerate}
		
		
		\textbf{Answer}
		
		\clearpage
		
		\section{Problem 7}
		Prove that the convergence of $\sum a_n$ implies the convergence of
		\[
		\sum \frac{\sqrt{a_n}}{n},
		\]
		if $a_n\ge 0$.
		
		\textbf{Answer}
		
		\clearpage
		
		\section{Problem 8} 
		If $\sum a_n$ converges, and if $\{b_n\}$ is monotonic and bounded,
		prove that $\sum a_n b_n$ converges.
		
		\textbf{Answer}
		
		\clearpage
		
		\section{Problem 9}
		Find the radius of convergence of each of the following power series:
		\begin{enumerate}
			\item[(a)] $\sum n^3 z^n,$
			\item[(b)] $\sum \dfrac{2^n}{n!} z^n,$
			\item[(c)] $\sum \dfrac{2^n}{n^2} z^n,$
			\item[(d)] $\sum \dfrac{n^3}{3^n} z^n.$
		\end{enumerate}
		
		\textbf{Answer}
		
		\clearpage
		
		\section{Problem 10}
		Suppose that the coefficients of the power series $\sum a_n z^n$ are integers, infinitely many of which are distinct from zero. Prove that the radius of convergence is at most $1$.
		
		\textbf{Answer}
		
		\clearpage
		
		\section{Problem 11} 
		Suppose $a_n > 0$, $s_n = a_1 + \cdots + a_n$, and $\sum a_n$ diverges.
		
		\begin{enumerate}
			\item[(a)] Prove that $\sum \dfrac{a_n}{1+a_n}$ diverges.
			
			\item[(b)] Prove that
			\[
			\frac{a_{N+1}}{s_{N+1}} + \cdots + \frac{a_{N+k}}{s_{N+k}}
			\ge 1 - \frac{s_N}{s_{N+k}}
			\]
			and deduce that $\sum \dfrac{a_n}{s_n}$ diverges.
			
			\item[(c)] Prove that
			\[
			\frac{a_n}{s_n^2} \le \frac{1}{s_{n-1}} - \frac{1}{s_n}
			\]
			and deduce that $\sum \dfrac{a_n}{s_n^2}$ converges.
			
			\item[(d)] What can be said about
			\[
			\sum \frac{a_n}{1+n a_n}
			\qquad \text{and} \qquad
			\sum \frac{a_n}{1+n^2 a_n}\, ?
			\]
		\end{enumerate}
	
		\textbf{Answer}
		
		\clearpage
		
		\section{Problem 12} 
		Suppose $a_n>0$ and $\sum a_n$ converges.
		Put
		\[
		r_n=\sum_{m=n}^{\infty} a_m.
		\]
		
		\begin{enumerate}
			\item[(a)] Prove that
			\[
			\frac{a_m}{r_m} + \cdots + \frac{a_n}{r_n} > 1 - \frac{r_n}{r_m}
			\]
			if $m<n$, and deduce that $\sum \dfrac{a_n}{r_n}$ diverges.
			
			\item[(b)] Prove that
			\[
			\frac{a_n}{\sqrt{r_n}} < 2(\sqrt{r_n} - \sqrt{r_{n+1}})
			\]
			and deduce that $\sum \dfrac{a_n}{\sqrt{r_n}}$ converges.
		\end{enumerate}
	
		\textbf{Answer}
		
		\clearpage
		
		\section{Problem 13} 
		Prove that the Cauchy product of two absolutely convergent series converges absolutely.
		
		
		\textbf{Answer}
		
		\clearpage
		
		\section{Problem 14} 
		If $\{s_n\}$ is a complex sequence, define its arithmetic means $\sigma_n$ by
		\[
		\sigma_n = \frac{s_0 + s_1 + \cdots + s_n}{n+1} \qquad (n=0,1,2,\dots).
		\]
		
		\begin{enumerate}
			\item[(a)] If $\lim s_n = s$, prove that $\lim \sigma_n = s$.
			
			\item[(b)] Construct a sequence $\{s_n\}$ which does not converge, although $\lim \sigma_n = 0$.
			
			\item[(c)] Can it happen that $s_n > 0$ for all $n$ and that $\limsup s_n = \infty$, although $\lim \sigma_n = 0$?
			
			\item[(d)] Put $a_n = s_n - s_{n-1}$, for $n \ge 1$. Show that
			\[
			s_n - \sigma_n = \frac{1}{n+1} \sum_{k=1}^{n} k a_k.
			\]
			Assume that $\lim(n a_n) = 0$ and that $\{\sigma_n\}$ converges.
			Prove that $\{s_n\}$ converges.
			[This gives a converse of (a), but under the additional assumption that $n a_n \to 0$.]
			
			\item[(e)] Derive the last conclusion from a weaker hypothesis:  
			Assume $M < \infty$, $|n a_n| \le M$ for all $n$, and $\lim\sigma_n = \sigma$.  
			Prove that $\lim s_n = \sigma$, by completing the following outline:
			
			If $m<n$, then
			\[
			s_n - \sigma_n
			= \frac{m+1}{n-m} (\sigma_n - \sigma_m)
			+ \frac{1}{n-m} \sum_{l=m+1}^{n} (s_n - s_l).
			\]
			
			For these $l$,
			\[
			|s_n - s_l| \le \frac{(n-l)M}{l+1} \le \frac{(n-m-1)M}{m+2}.
			\]
			
			Fix $\varepsilon > 0$ and associate with each $n$ the integer $m$ that satisfies
			\[
			m \le \frac{n-\varepsilon}{1+\varepsilon} < m+1.
			\]
			
			Then $(m+1)/(n-m) \le 1/\varepsilon$ and $|s_n - s_l| < M\varepsilon$. Hence
			\[
			\limsup_{n\to\infty}|s_n - \sigma| \le M\varepsilon.
			\]
			Since $\varepsilon$ was arbitrary, $\lim s_n = \sigma$.
		\end{enumerate}
			
		\textbf{Answer}
		
		\clearpage
		
		
		\section{Problem 15} 
		Definition 3.21 can be extended to the case in which the $a_n$ lie in some fixed $\mathbb{R}^k$.
		Absolute convergence is defined as convergence of $\sum |a_n|$. Show that Theorems
		3.22, 3.23, 3.25(a), 3.33, 3.34, 3.42, 3.45, 3.47, and 3.55 are true in this more
		general setting. (Only slight modifications are required in any of the proofs.)
		
		
		\textbf{Answer}
		
		\clearpage
		
		\section{Problem 16} 
		Fix a positive number $\alpha$. Choose $x_1 > \sqrt{\alpha}$, and define
		$x_2, x_3, x_4, \dots$ by the recursion formula
		\[
		x_{n+1} = \frac12\left( x_n + \frac{\alpha}{x_n} \right).
		\]
		
		\begin{enumerate}
			\item[(a)] Prove that $\{x_n\}$ decreases monotonically and that $\lim x_n = \sqrt{\alpha}$.
			
			\item[(b)] Put $\varepsilon_n = x_n - \sqrt{\alpha}$, and show that
			\[
			\varepsilon_{n+1}
			= \frac{\varepsilon_n^2}{2x_n}
			< \frac{\varepsilon_n^2}{2\sqrt{\alpha}},
			\]
			so that, setting $\beta = 2\sqrt{\alpha}$,
			\[
			\varepsilon_{n+1} < \beta \left( \frac{\varepsilon_1}{\beta} \right)^{2^n}
			\qquad(n=1,2,3,\dots).
			\]
			
			\item[(c)] This is a good algorithm for computing square roots, since the recursion
			formula is simple and the convergence is extremely rapid.  
			For example, if $\alpha = 3$ and $x_1 = 2$, show that  
			$\varepsilon_1/\beta < \frac{1}{10}$ and that therefore  
			\[
			\varepsilon_5 < 4\cdot 10^{-16}, \qquad
			\varepsilon_6 < 4\cdot 10^{-32}.
			\]
		\end{enumerate}
	
		\textbf{Answer}
		
		\clearpage
		
		\section{Problem 17}
		Fix $\alpha > 1$. Take $x_1 > \sqrt{\alpha}$, and define
		\[
		x_{n+1} = \frac{\alpha + x_n}{1 + x_n}
		= x_n + \frac{\alpha - x_n^2}{1 + x_n}.
		\]
		
		\begin{enumerate}
			\item[(a)] Prove that $x_1 > x_3 > x_5 > \cdots$.
			
			\item[(b)] Prove that $x_2 < x_4 < x_6 < \cdots$.
			
			\item[(c)] Prove that $\lim x_n = \sqrt{\alpha}$.
			
			\item[(d)] Compare the rapidity of convergence of this process with the one described in Exercise 16.
		\end{enumerate}
		
		\textbf{Answer}
		
		\clearpage
		
		\section{Problem 18}
		Replace the recursion formula of Exercise 16 by
		\[
		x_{n+1} = \frac{p-1}{p} x_n + \frac{\alpha}{p} x_n^{-p+1}
		\]
		where $p$ is a fixed positive integer, and describe the behavior of the resulting sequences $\{x_n\}$.
		
		
		\textbf{Answer}
		
		\clearpage
		
		\section{Problem 19} 
		Associate to each sequence $a = \{\alpha_n\}$, in which $\alpha_n$ is $0$ or $2$, the real number
		\[
		x(a) = \sum_{n=1}^{\infty} \frac{\alpha_n}{3^n}.
		\]
		Prove that the set of all $x(a)$ is precisely the Cantor set described in Sec. 2.44.
		
		
		\textbf{Answer}
		
		\clearpage
		
		\section{Problem 20} 
		Suppose $\{p_n\}$ is a Cauchy sequence in a metric space $X$, and some subsequence $\{p_{n_k}\}$
		converges to a point $p \in X$. Prove that the full sequence $\{p_n\}$ converges to $p$.
		
		
		\textbf{Answer}
		
		\clearpage
		
		\section{Problem 21} 
		Prove the following analogue of Theorem 3.10(b):  
		If $\{E_n\}$ is a sequence of closed nonempty and bounded sets in a complete metric space $X$,  
		if $E_n \supset E_{n+1}$, and if
		\[
		\lim_{n\to\infty} \operatorname{diam}(E_n) = 0,
		\]
		then $\bigcap_{n=1}^\infty E_n$ consists of exactly one point.
		
		
		\textbf{Answer}
		
		\clearpage
		
		\section{Problem 22} 
		Suppose $X$ is a nonempty complete metric space, and $\{G_n\}$ is a sequence of dense open subsets of $X$. Prove Baire’s theorem, namely, that $\bigcap_{n=1}^\infty G_n$ is not empty. (In fact, it is dense in $X$.)  
		
		\textit{Hint}: Find a shrinking sequence of neighborhoods $E_n$ such that $E_n \subset G_n$, and apply Exercise 21.
		
		
		\textbf{Answer}
		
		\clearpage
		
		\section{Problem 23} 
		Suppose $\{p_n\}$ and $\{q_n\}$ are Cauchy sequences in a metric space $X$.
		Show that the sequence $\{d(p_n,q_n)\}$ converges.  
		\textit{Hint:} For any $m,n$,
		\[
		d(p_n,q_n) \le d(p_n,p_m) + d(p_m,q_m) + d(q_m,q_n);
		\]
		it follows that
		\[
		|d(p_n,q_n) - d(p_m,q_m)|
		\]
		is small if $m$ and $n$ are large.
		
		\textbf{Answer}
		
		\clearpage
		
		\section{Problem 24} 
		Let $X$ be a metric space.
		
		\begin{enumerate}
			\item[(a)] Call two Cauchy sequences $\{p_n\}$, $\{q_n\}$ in $X$ equivalent if
			\[
			\lim_{n\to\infty} d(p_n,q_n) = 0.
			\]
			Prove that this is an equivalence relation.
			
			\item[(b)] Let $X^*$ be the set of all equivalence classes so obtained.
			If $P \in X^*$, $Q \in X^*$, $\{p_n\} \in P$, $\{q_n\} \in Q$, define
			\[
			\Delta(P,Q) = \lim_{n\to\infty} d(p_n,q_n);
			\]
			by Exercise 23, this limit exists.
			Show that the number $\Delta(P,Q)$ is unchanged if $\{p_n\}$ and $\{q_n\}$
			are replaced by equivalent sequences, and hence that $\Delta$ is a distance function in $X^*$.
		
		
		  \item[(c)] Prove that the resulting metric space $X^*$ is complete.
		
		\item[(d)] For each $p \in X$, there is a Cauchy sequence all of whose terms are $p$;
		let $P_p$ be the element of $X^*$ which contains this sequence.
		Prove that
		\[
		\Delta(P_p, P_q) = d(p,q)
		\]
		for all $p,q \in X$.  
		In other words, the mapping $\varphi$ defined by $\varphi(p) = P_p$ is an isometry  
		(i.e., a distance-preserving mapping) of $X$ into $X^*$.
		
		\item[(e)] Prove that $\varphi(X)$ is dense in $X^*$, and that $\varphi(X) = X^*$ if $X$ is complete.  
		By (d), we may identify $X$ and $\varphi(X)$ and thus regard $X$ as embedded in the complete
		metric space $X^*$.  
		We call $X^*$ the \emph{completion} of $X$.
	\end{enumerate}
	
		\textbf{Answer}
		
		\clearpage
		
		\section{Problem 25} 
		Let $X$ be the metric space whose points are the rational numbers,
		with the metric $d(x,y)=|x-y|$. What is the completion of this space? (Compare Exercise 24.)
		
		\textbf{Answer}
		
		\clearpage
		
		\end{document}
