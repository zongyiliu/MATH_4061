\documentclass[letterpaper]{article} 
\usepackage[utf8]{inputenc}
\usepackage[T1]{fontenc}
\usepackage{amsmath}
\usepackage{mathrsfs}
\usepackage{amsfonts}
\usepackage{amssymb}
\usepackage{array}
\usepackage{booktabs}
\usepackage{hyperref}
\usepackage[version=4]{mhchem}
\usepackage{stmaryrd}
\usepackage[dvipsnames]{xcolor}
\colorlet{LightRubineRed}{RubineRed!70}
\colorlet{Mycolor1}{green!10!orange}
\definecolor{Mycolor2}{HTML}{00F9DE}
\usepackage{graphicx}
\usepackage{amsmath}
\usepackage{graphicx}
\usepackage{capt-of}
\usepackage{lipsum}
\usepackage{fancyvrb}
\usepackage{tabularx}
\usepackage{listings}
\usepackage[export]{adjustbox}
\graphicspath{ {./images/} }
\usepackage[utf8]{inputenc}
\usepackage[english]{babel}
\usepackage{float}
\usepackage{lipsum}
\usepackage{graphicx}
\usepackage{float}
\usepackage[margin=0.7in]{geometry}
\usepackage{amsmath}
\usepackage{graphicx}
\usepackage{capt-of}
\usepackage{tcolorbox}
\usepackage{lipsum}
\usepackage{graphicx}
\usepackage{float}
\usepackage{listings}
\usepackage{hyperref} 
\usepackage{xcolor} % For custom colors
\lstset{
	language=Python,                % Choose the language (e.g., Python, C, R)
	basicstyle=\ttfamily\small, % Font size and type
	keywordstyle=\color{blue},  % Keywords color
	commentstyle=\color{gray},  % Comments color
	stringstyle=\color{red},    % String color
	numbers=left,               % Line numbers
	numberstyle=\tiny\color{gray}, % Line number style
	stepnumber=1,               % Numbering step
	breaklines=true,            % Auto line break
	backgroundcolor=\color{black!5}, % Light gray background
	frame=single,               % Frame around the code
}
\usepackage{float}
\usepackage[]{amsthm} %lets us use \begin{proof}
	\usepackage[]{amssymb} %gives us the character \varnothing
	
	\title{Exercise 6, MATH 4061}
	\author{Zongyi Liu}
	\date{Thu, Nov 27, 2025}
	
	\begin{document}
		\maketitle
		
		\section{Problem 1}
		
		Suppose $\alpha$ increases on $[a,b]$, $a \le x_0 \le b$, $\alpha$ is continuous at $x_0$, $f(x_0)=1$, and $f(x)=0$ if $x\ne x_0$. Prove that $f\in \mathcal{R}(\alpha)$ and that
		$\displaystyle \int f\,d\alpha = 0$.
		
		\textbf{Answer}
		
		\clearpage
		
		\section{Problem 2}
		
		Suppose $f \ge 0$, $f$ is continuous on $[a,b]$, and $\displaystyle \int_a^b f(x)\,dx = 0$. Prove that $f(x)=0$ for all $x\in [a,b]$. (Compare this with Exercise 1.)
		
		
		\textbf{Answer}
		
		\clearpage
		
		\section{Problem 3}
		
		Define three functions $\beta_1,\beta_2,\beta_3$ as follows: $\beta_j(x)=0$ if $x<0$, $\beta_j(x)=1$ if $x>0$ for $j=1,2,3$; and $\beta_1(0)=0$, $\beta_2(0)=1$, $\beta_3(0)=\tfrac12$. Let $f$ be a bounded function on $[-1,1]$.
		
		(a) Prove that $f\in \mathcal{R}(\beta_1)$ if and only if $f(0+)=f(0)$ and that then $\displaystyle \int f\,d\beta_1 = f(0)$.
		
		(b) State and prove a similar result for $\beta_2$.
		
		(c) Prove that $f\in \mathcal{R}(\beta_3)$ if and only if $f$ is continuous at $0$.
		
		(d) If $f$ is continuous at $0$ prove that
			$\displaystyle \int f\,d\beta_1 = \int f\,d\beta_2 = \int f\,d\beta_3 = f(0)$.

		
		\textbf{Answer}
		
		\clearpage
		
		\section{Problem 4}
		If $f(x)=0$ for all irrational $x$, $f(x)=1$ for all rational $x$, prove that $f\notin \mathcal{R}$ on $[a,b]$ for any $a<b$.
		
		
		\textbf{Answer}
		
		\clearpage
		
		\section{Problem 5}
		
		Suppose $f$ is a bounded real function on $[a,b]$, and $f^2 \in \mathcal{R}$ on $[a,b]$. Does it follow that $f\in \mathcal{R}$? Does the answer change if we assume that $f^3 \in \mathcal{R}$?
		
		
		\textbf{Answer}
		
		\clearpage
		
		\section{Problem 6}
		Let $P$ be the Cantor set constructed in Sec.\ 2.44. Let $f$ be a bounded real function on $[0,1]$ which is continuous at every point outside $P$. Prove that $f\in \mathcal{R}$ on $[0,1]$. \textit{Hint:} $P$ can be covered by finitely many segments whose total length can be made as small as desired. Proceed as in Theorem 6.10.
		
		
		\textbf{Answer}
		
		\clearpage
		
		\section{Problem 7}
		
		Suppose $f$ is a real function on $(0,1]$ and $f\in \mathcal{R}$ on $[c,1]$ for every $c>0$. Define
		\[
		\int_0^1 f(x)\,dx = \lim_{c\to 0} \int_c^1 f(x)\,dx
		\]
		if this limit exists (and is finite).
		
		\textbf{Answer}
		
		\clearpage
		
		\section{Problem 8}
		
		Suppose $f \in \mathscr{R}$ on $[a, b]$ for every $b>a$ where $a$ is fixed. Define
		
		$$
		\int_{a}^{\infty} f(x) d x=\lim _{b \rightarrow \infty} \int_{a}^{b} f(x) d x
		$$
		
		if this limit exists (and is finite). In that case, we say that the integral on the left converges. If it also converges after $f$ has been replaced by $|f|$, it is said to converge absolutely.
		
		Assume that $f(x) \geq 0$ and that $f$ decreases monotonically on $[1, \infty)$. Prove that
		
		$$
		\int_{1}^{\infty} f(x) d x
		$$
		
		converges if and only if
		
		$$
		\sum_{n=1}^{\infty} f(n)
		$$
		
		converges. (This is the so-called "integral test" for convergence of series.)
		
		\textbf{Answer}
		
		\clearpage
		
		\section{Problem 9}
		
		Show that integration by parts can sometimes be applied to the "improper" integrals defined in Exercises 7 and 8. (State appropriate hypotheses, formulate a theorem, and prove it.) For instance show that
		
		$$
		\int_{0}^{\infty} \frac{\cos x}{1+x} d x=\int_{0}^{\infty} \frac{\sin x}{(1+x)^{2}} d x
		$$
		
		Show that one of these integrals converges absolutely, but that the other does not.
		
		\textbf{Answer}
		
		\clearpage
		
		\section{Problem 10}
		
		
		Let $p$ and $q$ be positive real numbers such that
		
		$$
		\frac{1}{p}+\frac{1}{q}=1 .
		$$
		
		Prove the following statements.
		
		(a) If $u \geq 0$ and $v \geq 0$, then
		
		$$
		u v \leq \frac{u^{p}}{p}+\frac{v^{q}}{q}
		$$
		
		Equality holds if and only if $u^{p}=v^{q}$.
		
		(b) If $f \in \mathscr{R}(\alpha), g \in \mathscr{R}(\alpha), f \geq 0, g \geq 0$, and
		
		$$
		\int_{a}^{b} f^{p} d \alpha=1=\int_{a}^{b} g^{q} d \alpha
		$$
		
		then
		
		$$
		\int_{a}^{b} f g d \alpha \leq 1
		$$
		
		(c) If $f$ and $g$ are complex functions in $\mathscr{R}(\alpha)$, then
		
		$$
		\left|\int_{a}^{b} f g d \alpha\right| \leq\left\{\int_{a}^{b}|f|^{p} d \alpha\right\}^{1 / p}\left\{\int_{a}^{b}|g|^{a} d \alpha\right\}^{1 / a}
		$$
		
		This is Hölder's inequality. When $p=q=2$ it is usually called the Schwarz inequality. (Note that Theorem 1.35 is a very special case of this.)
	
		(d) Show that Hölder's inequality is also true for the "improper" integrals described in Exercises 7 and 8.
		
		\textbf{Answer}
		
		\clearpage
		
		\section{Problem 11} 
		
		Let $\alpha$ be a fixed increasing function on $[a, b]$. For $u \in \mathscr{R}(\alpha)$, define
		
		$$
		\|u\|_{2}=\left\{\int_{a}^{b}|u|^{2} d \alpha\right\}^{1 / 2}
		$$
		
		Suppose $f, g, h \in \mathscr{R}(\alpha)$, and prove the triangle inequality
		
		$$
		\|f-h\|_{2} \leq\|f-g\|_{2}+\|g-h\|_{2}
		$$
		
		as a consequence of the Schwarz inequality, as in the proof of Theorem 1.37.
		
		\textbf{Answer}
		
		\clearpage
		
		\section{Problem 12}
		
		With the notations of Exercise 11, suppose $f \in \mathscr{R}(\alpha)$ and $\varepsilon>0$. Prove that there exists a continuous function $g$ on $[a, b]$ such that $\|f-g\|_{2}<\varepsilon$.
		
		Hint: Let $P=\left\{x_{0}, \ldots, x_{n}\right\}$ be a suitable partition of $[a, b]$, define
		
		$$
		g(t)=\frac{x_{i}-t}{\Delta x_{i}} f\left(x_{i-1}\right)+\frac{t-x_{i-1}}{\Delta x_{i}} f\left(x_{i}\right)
		$$
		
		if $x_{i-1} \leq t \leq x_{i}$.
		
		\textbf{Answer}
		
		\clearpage
		
		\section{Problem 13}
		
		Define
		
		$$
		f(x)=\int_{x}^{x+1} \sin \left(t^{2}\right) d t
		$$
		
		(a) Prove that $|f(x)|<1 / x$ if $x>0$.
		
		Hint: Put $t^{2}=u$ and integrate by parts, to show that $f(x)$ is equal to
		
		$$
		\frac{\cos \left(x^{2}\right)}{2 x}-\frac{\cos \left[(x+1)^{2}\right]}{2(x+1)}-\int_{x^{2}}^{(x+1)^{2}} \frac{\cos u}{4 u^{3 / 2}} d u
		$$
		
		Replace $\cos u$ by -1.
		
		(b) Prove that
		
		$$
		2 x f(x)=\cos \left(x^{2}\right)-\cos \left[(x+1)^{2}\right]+r(x)
		$$
		
		where $|r(x)|<c / x$ and $c$ is a constant.
		
		(c) Find the upper and lower limits of $x f(x)$, as $x \rightarrow \infty$.
		
		(d) Does $\int_{0}^{\infty} \sin \left(t^{2}\right) d t$ converge?
		
		\textbf{Answer}
		
		\clearpage
		
		\section{Problem 14}
		
		Deal similarly with
		
		$$
		f(x)=\int_{x}^{x+1} \sin \left(e^{t}\right) d t
		$$
		
		Show that
		
		$$
		e^{x}|f(x)|<2
		$$
		
		and that
		
		$$
		e^{x} f(x)=\cos \left(e^{x}\right)-e^{-1} \cos \left(e^{x+1}\right)+r(x)
		$$
		
		where $|r(x)|<C e^{-x}$, for some constant $C$.
		
		\textbf{Answer}
		
		\clearpage
		
		\section{Problem 15}
		
		Suppose $f$ is a real, continuously differentiable function on $[a, b], f(a)=f(b)=0$, and
		
		$$
		\int_{a}^{b} f^{2}(x) d x=1
		$$
		
		Prove that
		
		$$
		\int_{a}^{b} x f(x) f^{\prime}(x) d x=-\frac{1}{2}
		$$
		
		and that
		
		$$
		\int_{a}^{b}\left[f^{\prime}(x)\right]^{2} d x \cdot \int_{a}^{b} x^{2} f^{2}(x) d x>4
		$$
		
		\textbf{Answer}
		
		\clearpage
		
		\section{Problem 16}
		
		For $1<s<\infty$, define

$$
\zeta(s)=\sum_{n=1}^{\infty} \frac{1}{n^{s}} .
$$

(This is Riemann's zeta function, of great importance in the study of the distribution of prime numbers.) Prove that

(a) $\zeta(s)=s \int_{1}^{\infty} \frac{[x]}{x^{s+1}} d x$ and that

(b) $\zeta(s)=\frac{s}{s-1}-s \int_{1}^{\infty} \frac{x-[x]}{x^{s+1}} d x$, where $[x]$ denotes the greatest integer $\leq x$. Prove that the integral in $(b)$ converges for all $s>0$.

\emph{Hint}: To prove $(a)$, compute the difference between the integral over $[1, N]$ and the $N$ th partial sum of the series that defines $\zeta(s)$.\
		
		\textbf{Answer}
		
		\clearpage
		
		\section{Problem 17}
		
		Suppose $\alpha$ increases monotonically on $[a, b], g$ is continuous, and $g(x)=G^{\prime}(x)$ for $a \leq x \leq b$. Prove that
		
		$$
		\int_{a}^{b} \alpha(x) g(x) d x=G(b) \alpha(b)-G(a) \alpha(a)-\int_{a}^{b} G d \alpha
		$$
		
		\emph{Hint}: Take $g$ real, without loss of generality. Given $P=\left\{x_{0}, x_{1}, \ldots, x_{n}\right\}$, choose $t_{i} \in\left(x_{i-1}, x_{i}\right)$ so that $g\left(t_{i}\right) \Delta x_{i}=G\left(x_{i}\right)-G\left(x_{i-1}\right)$. Show that
		
		$$
		\sum_{i=1}^{n} \alpha\left(x_{i}\right) g\left(t_{i}\right) \Delta x_{i}=G(b) \alpha(b)-G(a) \alpha(a)-\sum_{i=1}^{n} G\left(x_{i-1}\right) \Delta \alpha_{i} .
		$$
		
		\textbf{Answer}
		
		\clearpage
		
		\section{Problem 18}
		
		Let $\gamma_{1}, \gamma_{2}, \gamma_{3}$ be curves in the complex plane, defined on $[0,2 \pi]$ by

$$
\gamma_{1}(t)=e^{i t}, \quad \gamma_{2}(t)=e^{2 i t}, \quad \gamma_{3}(t)=e^{2 \pi i t \sin (1 / t)} .
$$

Show that these three curves have the same range, that $\gamma_{1}$ and $\gamma_{2}$ are rectifiable, that the length of $\gamma_{1}$ is $2 \pi$, that the length of $\gamma_{2}$ is $4 \pi$, and that $\gamma_{3}$ is not rectifiable.
		
		\textbf{Answer}
		
		\clearpage
		
		\section{Problem 19}
		
		Let $\gamma_{1}$ be a curve in $R^{k}$, defined on $[a, b]$; let $\phi$ be a continuous 1-1 mapping of $[c, d]$ onto $[a, b]$, such that $\phi(c)=a$; and define $\gamma_{2}(s)=\gamma_{1}(\phi(s))$. Prove that $\gamma_{2}$ is an arc, a closed curve, or a rectifiable curve if and only if the same is true of $\gamma_{1}$. Prove that $\gamma_{2}$ and $\gamma_{1}$ have the same length.
		
		
		\textbf{Answer}
		
		\clearpage
		
	\end{document}
	
	
