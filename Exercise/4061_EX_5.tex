\documentclass[letterpaper]{article} 
\usepackage[utf8]{inputenc}
\usepackage[T1]{fontenc}
\usepackage{amsmath}
\usepackage{amsfonts}
\usepackage{amssymb}
\usepackage{array}
\usepackage{booktabs}
\usepackage{hyperref}
\usepackage[version=4]{mhchem}
\usepackage{stmaryrd}
\usepackage[dvipsnames]{xcolor}
\colorlet{LightRubineRed}{RubineRed!70}
\colorlet{Mycolor1}{green!10!orange}
\definecolor{Mycolor2}{HTML}{00F9DE}
\usepackage{graphicx}
\usepackage{amsmath}
\usepackage{graphicx}
\usepackage{capt-of}
\usepackage{lipsum}
\usepackage{fancyvrb}
\usepackage{tabularx}
\usepackage{listings}
\usepackage[export]{adjustbox}
\graphicspath{ {./images/} }
\usepackage[utf8]{inputenc}
\usepackage[english]{babel}
\usepackage{float}
\usepackage{lipsum}
\usepackage{graphicx}
\usepackage{float}
\usepackage[margin=0.7in]{geometry}
\usepackage{amsmath}
\usepackage{graphicx}
\usepackage{capt-of}
\usepackage{tcolorbox}
\usepackage{lipsum}
\usepackage{graphicx}
\usepackage{float}
\usepackage{listings}
\usepackage{hyperref} 
\usepackage{xcolor} % For custom colors
\lstset{
	language=Python,                % Choose the language (e.g., Python, C, R)
	basicstyle=\ttfamily\small, % Font size and type
	keywordstyle=\color{blue},  % Keywords color
	commentstyle=\color{gray},  % Comments color
	stringstyle=\color{red},    % String color
	numbers=left,               % Line numbers
	numberstyle=\tiny\color{gray}, % Line number style
	stepnumber=1,               % Numbering step
	breaklines=true,            % Auto line break
	backgroundcolor=\color{black!5}, % Light gray background
	frame=single,               % Frame around the code
}
\usepackage{float}
\usepackage[]{amsthm} %lets us use \begin{proof}
\usepackage[]{amssymb} %gives us the character \varnothing

	\title{Exercise 5, MATH 4061}
	\author{Zongyi Liu}
	\date{Fri, Nov 21, 2025}
	
	\begin{document}
		\maketitle
		\section{Problem 1} 
		
		Let $f$ be defined for all real $x$, and suppose that

$$
|f(x)-f(y)| \leq(x-y)^{2}
$$

for all real $x$ and $y$. Prove that $f$ is constant.
		
		\textbf{Answer}
		
		\clearpage
		
		\section{Problem 2} 
		
		Suppose $f^{\prime}(x)>0$ in $(a, b)$. Prove that $f$ is strictly increasing in $(a, b)$, and let $g$ be its inverse function. Prove that $g$ is differentiable, and that
		
		$$
		g^{\prime}(f(x))=\frac{1}{f^{\prime}(x)} \quad(a<x<b)
		$$
		
		\textbf{Answer}
		
		\clearpage
		
		\section{Problem 3} 
		
		Suppose $g$ is a real function on $R^{1}$, with bounded derivative (say $\left|g^{\prime}\right| \leq M$). Fix $\varepsilon>0$, and define $f(x)=x+\varepsilon g(x)$. Prove that $f$ is one-to-one if $\varepsilon$ is small enough. (A set of admissible values of $\varepsilon$ can be determined which depends only on $M$.)
		
		\textbf{Answer}
		
		\clearpage
		
		\section{Problem 4} 
		
		If

$$
C_{0}+\frac{C_{1}}{2}+\cdots+\frac{C_{n-1}}{n}+\frac{C_{n}}{n+1}=0
$$

where $C_{0}, \ldots, C_{n}$ are real constants, prove that the equation

$$
C_{0}+C_{1} x+\cdots+C_{n-1} x^{n-1}+C_{n} x^{n}=0
$$

has at least one real root between 0 and 1.
		
		\textbf{Answer}
		
		\clearpage
		
		\section{Problem 5} 
		
		Suppose $f$ is defined and differentiable for every $x>0$, and $f^{\prime}(x) \rightarrow 0$ as $x \rightarrow+\infty$. Put $g(x)=f(x+1)-f(x)$. Prove that $g(x) \rightarrow 0$ as $x \rightarrow+\infty$.
		
		\textbf{Answer}
		
		\clearpage
		
		\section{Problem 6} 
		
		Suppose
		
		(a) $f$ is continuous for $x \geq 0$,
		
		(b) $f^{\prime}(x)$ exists for $x>0$,
		
		(c) $f(0)=0$,
		
		(d) $f^{\prime}$ is monotonically increasing. Put
		
		$$
		g(x)=\frac{f(x)}{x} \quad(x>0)
		$$
		
		and prove that $g$ is monotonically increasing.
		
		\textbf{Answer}
		
		\clearpage
		
		\section{Problem 7} 
		
		Suppose $f^{\prime}(x), g^{\prime}(x)$ exist, $g^{\prime}(x) \neq 0$, and $f(x)=g(x)=0$. Prove that
		
		$$
		\lim _{t \rightarrow x} \frac{f(t)}{g(t)}=\frac{f^{\prime}(x)}{g^{\prime}(x)}
		$$
		
		(This holds also for complex functions.)
		
		\textbf{Answer}
		
		\clearpage
		
		\section{Problem 8} 
		
		Suppose $f^{\prime}$ is continuous on $[a, b]$ and $\varepsilon>0$. Prove that there exists $\delta>0$ such that
		
		$$
		\left|\frac{f(t)-f(x)}{t-x}-f^{\prime}(x)\right|<\varepsilon
		$$
		
		whenever $0<|t-x|<\delta, a \leq x \leq b, a \leq t \leq b$. (This could be expressed by saying that $f$ is uniformly differentiable on $[a, b]$ if $f^{\prime}$ is continuous on $[a, b]$.) Does this hold for vector-valued functions too?
		
		\textbf{Answer}
		
		\clearpage
		
		\section{Problem 9} 
		
		Let $f$ be a continuous real function on $R^{1}$, of which it is known that $f^{\prime}(x)$ exists for all $x \neq 0$ and that $f^{\prime}(x) \rightarrow 3$ as $x \rightarrow 0$. Does it follow that $f^{\prime}(0)$ exists?
		
		\textbf{Answer}
		
		\clearpage
		
		\section{Problem 10} 
		
		Suppose $f$ and $g$ are complex differentiable functions on $(0,1), f(x) \rightarrow 0, g(x) \rightarrow 0$, $f^{\prime}(x) \rightarrow A, g^{\prime}(x) \rightarrow B$ as $x \rightarrow 0$, where $A$ and $B$ are complex numbers, $B \neq 0$. Prove that
		
		$$
		\lim _{x \rightarrow 0} \frac{f(x)}{g(x)}=\frac{A}{B}
		$$
		
		Compare with Example 5.18. \emph{Hint}:
		
		$$
		\frac{f(x)}{g(x)}=\left\{\frac{f(x)}{x}-A\right\} \cdot \frac{x}{g(x)}+A \cdot \frac{x}{g(x)} .
		$$
		
		Apply Theorem 5.13 to the real and imaginary parts of $f(x) / x$ and $g(x) / x$.
		
		\textbf{Answer}
		
		\clearpage
		
		\section{Problem 11} 
		
		Suppose $f$ is defined in a neighborhood of $x$, and suppose $f^{\prime \prime}(x)$ exists. Show that
		
		$$
		\lim _{h \rightarrow 0} \frac{f(x+h)+f(x-h)-2 f(x)}{h^{2}}=f^{\prime \prime}(x)
		$$
		
		Show by an example that the limit may exist even if $f^{\prime \prime}(x)$ does not.
		
		\emph{Hint}: Use Theorem 5.13.
		
		\textbf{Answer}
		
		\clearpage
		
		\section{Problem 12} 
		
		If $f(x)=|x|^{3}$, compute $f^{\prime}(x), f^{\prime \prime}(x)$ for all real $x$, and show that $f^{(3)}(0)$ does not exist.
		
		\textbf{Answer}
		
		\clearpage
		
		\section{Problem 13}
		
		Suppose $a$ and $c$ are real numbers, $c>0$, and $f$ is defined on $[-1,1]$ by
		
		$$
		f(x)= \begin{cases}x^{a} \sin \left(|x|^{-c}\right) & (\text { if } x \neq 0) \\ 0 & (\text { if } x=0)\end{cases}
		$$
		
		Prove the following statements:\\
		(a) ${f}$ is continuous if and only if ${a}>{0}$.\\
		(b) $f^{\prime}(0)$ exists if and only if $a>1$.\\
		(c) $f^{\prime}$ is bounded if and only if $a \geq 1+c$.\\
		(d) $f^{\prime}$ is continuous if and only if $a>1+c$.\\
		(e) $f^{\prime \prime}(0)$ exists if and only if $a>2+c$.\\
		(f) $f^{\prime \prime}$ is bounded if and only if $a \geq 2+2 c$.\\
		(g) $f^{\prime \prime}$ is continuous if and only if $a>2+2 c$.
		
		\textbf{Answer}
		
		\clearpage
		
		\section{Problem 14} 
		
		Let $f$ be a differentiable real function defined in ($a, b$). Prove that $f$ is convex if and only if $f^{\prime}$ is monotonically increasing. Assume next that $f^{\prime \prime}(x)$ exists for every $x \in(a, b)$, and prove that $f$ is convex if and only if $f^{\prime \prime}(x) \geq 0$ for all $x \in(a, b)$.
		
		\textbf{Answer}
		
		\clearpage
		
		\section{Problem 15} 
		
		Suppose $a \in R^{1}, f$ is a twice-differentiable real function on ($a, \infty$), and $M_{0}, M_{1}, M_{2}$ are the least upper bounds of $|f(x)|,\left|f^{\prime}(x)\right|,\left|f^{\prime \prime}(x)\right|$, respectively, on $(a, \infty)$. Prove that
		
		$$
		M_{1}^{2} \leq 4 M_{0} M_{2}
		$$
		
		Hint: If $h>0$, Taylor's theorem shows that
		
		$$
		f^{\prime}(x)=\frac{1}{2 h}[f(x+2 h)-f(x)]-h f^{\prime \prime}(\xi)
		$$
		
		for some $\xi \in(x, x+2 h)$. Hence
		
		$$
		\left|f^{\prime}(x)\right| \leq h M_{2}+\frac{M_{0}}{h}
		$$
		
		To show that $M_{1}^{2}=4 M_{0} M_{2}$ can actually happen, take $a=-1$, define
		
		$$
		f(x)= \begin{cases}2 x^{2}-1 & (-1<x<0) \\ \frac{x^{2}-1}{x^{2}+1} & (0 \leq x<\infty)\end{cases}
		$$
		
		and show that $M_{0}=1, M_{1}=4, M_{2}=4$.\\
		Does $M_{1}^{2} \leq 4 M_{0} M_{2}$ hold for vector-valued function too?
		
		\textbf{Answer}
		
		\clearpage
		
		\section{Problem 16}
		
		Suppose $f$ is twice-differentiable on $(0, \infty), f^{\prime \prime}$ is bounded on $(0, \infty)$, and $f(x) \rightarrow 0$ as $x \rightarrow \infty$. Prove that $f^{\prime}(x) \rightarrow 0$ as $x \rightarrow \infty$.
		
		Hint: Let $a \rightarrow \infty$ in Exercise 15.
		
		\textbf{Answer}
		
		\clearpage
		
		\section{Problem 17}
		
		Suppose $f$ is a real, three times differentiable function on $[-1,1]$, such that
		
		$$
		f(-1)=0, \quad f(0)=0, \quad f(1)=1, \quad f^{\prime}(0)=0
		$$
		
		Prove that $f^{(3)}(x) \geq 3$ for some $x \in(-1,1)$. Note that equality holds for $\frac{1}{2}\left(x^{3}+x^{2}\right)$. \emph{Hint}: Use Theorem 5.15, with $\alpha=0$ and $\beta= \pm 1$, to show that there exist $s \in(0,1)$ and $t \in(-1,0)$ such that
		
		$$
		f^{(3)}(s)+f^{(3)}(t)=6
		$$
		
		\textbf{Answer}
		
		\clearpage
		
		\section{Problem 18}
		
		Suppose $f$ is a real function on $[a, b], n$ is a positive integer, and $f^{(n-1)}$ exists for every $t \in[a, b]$. Let $\alpha, \beta$, and $P$ be as in Taylor's theorem (5.15). Define:

$$
Q(t)=\frac{f(t)-f(\beta)}{t-\beta}
$$

for $t \in[a, b], t \neq \beta$, differentiate

$$
f(t)-f(\beta)=(t-\beta) Q(t)
$$

$n-1$ times at $t=\alpha$, and derive the following version of Taylor's theorem:

$$
f(\beta)=P(\beta)+\frac{Q^{(n-1)}(\alpha)}{(n-1)!}(\beta-\alpha)^{n}
$$

		
		\textbf{Answer}
		
		\clearpage
		
		\section{Problem 19}
		
		Suppose $f$ is defined in $(-1,1)$ and $f^{\prime}(0)$ exists. Suppose $-1<\alpha_{n}<\beta_{n}<1$, $\alpha_{n} \rightarrow 0$, and $\beta_{n} \rightarrow 0$ as $n \rightarrow \infty$. Define the difference quotients

$$
D_{n}=\frac{f\left(\beta_{n}\right)-f\left(\alpha_{n}\right)}{\beta_{n}-\alpha_{n}}
$$

Prove the following statements:

(a) If $\alpha_{n}<0<\beta_{n}$, then $\lim D_{n}=f^{\prime}(0)$.

(b) If $0<\alpha_{n}<\beta_{n}$ and $\left\{\beta_{n} /\left(\beta_{n}-\alpha_{n}\right)\right\}$ is bounded, then $\lim D_{n}=f^{\prime}(0)$.

(c) If $f^{\prime}$ is continuous in $(-1,1)$, then $\lim D_{n}=f^{\prime}(0)$.

Give an example in which $f$ is differentiable in $(-1,1)$ (but $f^{\prime}$ is not continuous at 0 ) and in which $\alpha_{n}, \beta_{n}$ tend to 0 in such a way that $\lim D_{n}$ exists but is different from $f^{\prime}(0)$.
		
		\textbf{Answer}
		
		\clearpage
		
		\section{Problem 20}
		
		Formulate and prove an inequality which follows from Taylor's theorem and which remains valid for vector-valued functions.
		
		\textbf{Answer}
		
		\clearpage
		
		\section{Problem 21}
		
		Let $E$ be a closed subset of $R^{1}$. We saw in Exercise 22, Chap. 4, that there is a real continuous function $f$ on $R^{1}$ whose zero set is $E$. Is it possible, for each closed set $E$, to find such an $f$ which is differentiable on $R^{1}$, or one which is $n$ times differentiable, or even one which has derivatives of all orders on $R^{1}$?
		
		\textbf{Answer}
		
		\clearpage
		
		\section{Problem 22}
		
		Suppose $f$ is a real function on $(-\infty, \infty)$. Call $x$ a fixed point of $f$ if $f(x)=x$. 
		
		(a) If $f$ is differentiable and $f^{\prime}(t) \neq 1$ for every real $t$, prove that $f$ has at most one fixed point.
		
		
		(b) Show that the function $f$ defined by
		
		$$
		f(t)=t+\left(1+e^{t}\right)^{-1}
		$$
		
		has no fixed point, although $0<f^{\prime}(t)<1$ for all real $t$.
		
		(c) However, if there is a constant $A<1$ such that $\left|f^{\prime}(t)\right| \leq A$ for all real $t$, prove that a fixed point $x$ of $f$ exists, and that $x=\lim x_{n}$, where $x_{1}$ is an arbitrary real number and
		
		$$
		x_{n+1}=f\left(x_{n}\right)
		$$
		
		for $n=1,2,3, \ldots$.
		
		(d) Show that the process described in (c) can be visualized by the zig-zag path
		
		$$
		\left(x_{1}, x_{2}\right) \rightarrow\left(x_{2}, x_{2}\right) \rightarrow\left(x_{2}, x_{3}\right) \rightarrow\left(x_{3}, x_{3}\right) \rightarrow\left(x_{3}, x_{4}\right) \rightarrow \cdots .
		$$
		
		
		\textbf{Answer}
		
		\clearpage
		
		\section{Problem 23}
		
		The function $f$ defined by

$$
f(x)=\frac{x^{3}+1}{3}
$$

has three fixed points, say $\alpha, \beta, \gamma$, where

$$
-2<\alpha<-1, \quad 0<\beta<1, \quad 1<\gamma<2 .
$$

For arbitrarily chosen $x_{1}$, define $\left\{x_{n}\right\}$ by setting $x_{n+1}=f\left(x_{n}\right)$.

(a) If $x_{1}<\alpha$, prove that $x_{n} \rightarrow-\infty$ as $n \rightarrow \infty$.

(b) If $\alpha<x_{1}<\gamma$, prove that $x_{n} \rightarrow \beta$ as $n \rightarrow \infty$.

(c) If $\gamma<x_{1}$, prove that $x_{n} \rightarrow+\infty$ as $n \rightarrow \infty$.

Thus $\beta$ can be located by this method, but $\alpha$ and $\gamma$ cannot.
		
		\textbf{Answer}
		
		\clearpage
		
		\section{Problem 24} 
		The process described in part (c) of Exercise 22 can of course also be applied to functions that map $(0, \infty)$ to $(0, \infty)$.
		
		Fix some $\alpha>1$, and put
		
		$$
		f(x)=\frac{1}{2}\left(x+\frac{\alpha}{x}\right), \quad g(x)=\frac{\alpha+x}{1+x} .
		$$
		
		Both $f$ and $g$ have $\sqrt{\alpha}$ as their only fixed point in $(0, \infty)$. Try to explain, on the basis of properties of $f$ and $g$, why the convergence in Exercise 16, Chap. 3, is so much more rapid than it is in Exercise 17. (Compare $f^{\prime}$ and $g^{\prime}$, draw the zig-zags suggested in Exercise 22.)
		
		Do the same when $0<\alpha<1$.
		
		\textbf{Answer}
		
		\clearpage
		
		\section{Problem 25}
		
		Suppose $f$ is twice differentiable on $[a, b], f(a)<0, f(b)>0, f^{\prime}(x) \geq \delta>0$, and $0 \leq f^{\prime \prime}(x) \leq M$ for all $x \in[a, b]$. Let $\xi$ be the unique point in $(a, b)$ at which $f(\xi)=0$.
		
		Complete the details in the following outline of Newton's method for computing $\xi$.
		
		(a) Choose $x_{1} \in(\xi, b)$, and define $\left\{x_{n}\right\}$ by
		
		$$
		x_{n+1}=x_{n}-\frac{f\left(x_{n}\right)}{f^{\prime}\left(x_{n}\right)}
		$$
		
		Interpret this geometrically, in terms of a tangent to the graph of $f$.
		
		(b) Prove that $x_{n+1}<x_{n}$ and that
		
		$$
		\lim _{n \rightarrow \infty} x_{n}=\xi
		$$
		
		(c) Use Taylor's theorem to show that
		
		$$
		x_{n+1}-\xi=\frac{f^{\prime \prime}\left(t_{n}\right)}{2 f^{\prime}\left(x_{n}\right)}\left(x_{n}-\xi\right)^{2}
		$$
		
		for some $t_{n} \in\left(\xi, x_{n}\right)$.
		
		(d) If $A=M / 2 \delta$, deduce that
		
		$$
		0 \leq x_{n+1}-\xi \leq \frac{1}{A}\left[A\left(x_{1}-\xi\right)\right]^{2 n}
		$$
		
		(Compare with Exercises 16 and 18, Chap. 3.)
		
		(e) Show that Newton's method amounts to finding a fixed point of the function $g$ defined by
		
		$$
		g(x)=x-\frac{f(x)}{f^{\prime}(x)}
		$$
		
		How does $g^{\prime}(x)$ behave for $x$ near $\xi$?
		
		(f) Put $f(x)=x^{1 / 3}$ on $(-\infty, \infty)$ and try Newton's method. What happens?
		
		\textbf{Answer}
		
		\clearpage
		
		\section{Problem 26}
		
		Suppose $f$ is differentiable on $[a, b], f(a)=0$, and there is a real number $A$ such that $\left|f^{\prime}(x)\right| \leq A|f(x)|$ on $[a, b]$. Prove that $f(x)=0$ for all $x \in[a, b]$. \emph{Hint}: Fix $x_{0} \in[a, b]$, let
		
		$$
		M_{0}=\sup |f(x)|, \quad M_{1}=\sup \left|f^{\prime}(x)\right|
		$$
		
		for $a \leq x \leq x_{0}$. For any such $x$,
		
		$$
		|f(x)| \leq M_{1}\left(x_{0}-a\right) \leq A\left(x_{0}-a\right) M_{0} .
		$$
		
		Hence $M_{0}=0$ if $A\left(x_{0}-a\right)<1$. That is, $f=0$ on $\left[a, x_{0}\right]$. Proceed.
		
		\textbf{Answer}
		
		\clearpage
		
		\section{Problem 27} 
		
		Let $\phi$ be a real function defined on a rectangle $R$ in the plane, given by $a \leq x \leq b$, $\alpha \leq y \leq \beta$. A solution of the initial-value problem
		
		$$
		y^{\prime}=\phi(x, y), \quad y(a)=c \quad(\alpha \leq c \leq \beta)
		$$
		
		is, by definition, a differentiable function $f$ on $[a, b]$ such that $f(a)=c, \alpha \leq f(x) \leq \beta$, and
		
		$$
		f^{\prime}(x)=\phi(x, f(x)) \quad(a \leq x \leq b) .
		$$
		
		Prove that such a problem has at most one solution if there is a constant $A$ such that
		
		$$
		\left|\phi\left(x, y_{2}\right)-\phi\left(x, y_{1}\right)\right| \leq A\left|y_{2}-y_{1}\right|
		$$
		
		whenever $\left(x, y_{1}\right) \in R$ and $\left(x, y_{2}\right) \in R$.\\
		Hint: Apply Exercise 26 to the difference of two solutions. Note that this uniqueness theorem does not hold for the initial-value problem
		
		$$
		y^{\prime}=y^{1 / 2}, \quad y(0)=0,
		$$
		
		which has two solutions: $f(x)=0$ and $f(x)=x^{2} / 4$. Find all other solutions.
		
		\textbf{Answer}
		
		\clearpage
		
		\section{Problem 28}
		
		Formulate and prove an analogous uniqueness theorem for systems of differential equations of the form
		
		$$
		y_{j}^{\prime}=\phi_{j}\left(x, y_{1}, \ldots, y_{k}\right), \quad y_{j}(a)=c_{j} \quad(j=1, \ldots, k)
		$$
		
		Note that this can be rewritten in the form
		
		$$
		\mathbf{y}^{\prime}=\boldsymbol{\phi}(x, \mathbf{y}), \quad \mathbf{y}(a)=\mathbf{c}
		$$
		
		where $\mathbf{y}=\left(y_{1}, \ldots, y_{k}\right)$ ranges over a $k$-cell, $\boldsymbol{\phi}$ is the mapping of a $(k+1)$-cell into the Euclidean $k$-space whose components are the functions $\phi_{1}, \ldots, \phi_{k}$, and $\mathbf{c}$ is the vector $\left(c_{1}, \ldots, c_{k}\right)$. Use Exercise 26, for vector-valued functions.
		
		\textbf{Answer}
		
		\clearpage
		
		\section{Problem 29} 
		
		Specialize Exercise 28 by considering the system
		
		$$
		\begin{aligned}
			y_{j}^{\prime} & =y_{j+1} \quad(j=1, \ldots, k-1) \\
			y_{k}^{\prime} & =f(x)-\sum_{j=1}^{k} g_{j}(x) y_{j}
		\end{aligned}
		$$
		
		where $f, g_{1}, \ldots, g_{k}$ are continuous real functions on $[a, b]$, and derive a uniqueness theorem for solutions of the equation
		
		$$
		y^{(k)}+g_{k}(x) y^{(k-1)}+\cdots+g_{2}(x) y^{\prime}+g_{1}(x) y=f(x)
		$$
		
		subject to initial conditions
		
		$$
		y(a)=c_{1}, \quad y^{\prime}(a)=c_{2}, \quad \ldots, \quad y^{(k-1)}(a)=c_{k}
		$$
		
		\textbf{Answer}
		
		\clearpage
		
	
		
		\end{document}
		
		
	
