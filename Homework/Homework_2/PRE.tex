\documentclass[letterpaper]{article} 
\usepackage[utf8]{inputenc}
\usepackage[T1]{fontenc}
\usepackage{amsmath}
\usepackage{amsfonts}
\usepackage{amssymb}
\usepackage{array}
\usepackage{booktabs}
\usepackage{hyperref}
\usepackage[version=4]{mhchem}
\usepackage{stmaryrd}
\usepackage[dvipsnames]{xcolor}
\colorlet{LightRubineRed}{RubineRed!70}
\colorlet{Mycolor1}{green!10!orange}
\definecolor{Mycolor2}{HTML}{00F9DE}
\usepackage{graphicx}
\usepackage{amsmath}
\usepackage{graphicx}
\usepackage{capt-of}
\usepackage{lipsum}
\usepackage{fancyvrb}
\usepackage{tabularx}
\usepackage{listings}
\usepackage[export]{adjustbox}
\graphicspath{ {./images/} }
\usepackage[utf8]{inputenc}
\usepackage[english]{babel}
\usepackage{float}
\usepackage{lipsum}
\usepackage{graphicx}
\usepackage{float}
\usepackage[margin=0.7in]{geometry}
\usepackage{amsmath}
\usepackage{graphicx}
\usepackage{capt-of}
\usepackage{tcolorbox}
\usepackage{lipsum}
\usepackage{graphicx}
\usepackage{float}
\usepackage{listings}
\usepackage{hyperref} 
\usepackage{xcolor} % For custom colors
\lstset{
	language=Python,                % Choose the language (e.g., Python, C, R)
	basicstyle=\ttfamily\small, % Font size and type
	keywordstyle=\color{blue},  % Keywords color
	commentstyle=\color{gray},  % Comments color
	stringstyle=\color{red},    % String color
	numbers=left,               % Line numbers
	numberstyle=\tiny\color{gray}, % Line number style
	stepnumber=1,               % Numbering step
	breaklines=true,            % Auto line break
	backgroundcolor=\color{black!5}, % Light gray background
	frame=single,               % Frame around the code
}
\usepackage{float}
\usepackage[]{amsthm} %lets us use \begin{proof}
\usepackage[]{amssymb} %gives us the character \varnothing

	\title{Homework 2, MATH 4061}
	\author{Zongyi Liu}
	\date{Mon, Sept 22, 2025}
	\begin{document}
		\maketitle
		
		\section{Question 1}
		\textbf{ Complex numbers}
		
		
		Show that $\mathbb{C}$ is a field, i.e. show that $\mathbb{C}$ satisfies axioms of addition (A1), (A2), (A3), (A4), the axioms of multiplication (M1), (M2), (M3), (M4), and the axiom of distribution (D).
		
	
		\textbf{Answer}
		
		\clearpage
		
		\section{Question 2}
		
		\textbf{Ordered fields}
		
		Prove that there is no order relation on $\mathbb{C}$ to turn $\mathbb{C}$ into an ordered field, i.e. there exists no order relation on $\mathbb{C}$ which satisfies axioms (O1), (O2), (O3), (O4), (OC1), and (OC2).
		
	\textbf{Answer}
	
			\clearpage
	
	\section{Question 3}
	
	\textbf{Absolute value of complex numbers}
	
	Show that the following statements hold:
	
	\begin{enumerate}
		\item Positive definiteness: $\forall z \in \mathbb{C}:|z| \geq 0$ and $|z|=0 \Longleftrightarrow z=0$,
		\item Multiplicativity: $\forall z, w \in \mathbb{C}:|z \cdot w|=|z| \cdot|w|$,
		\item Triangle inequality: $\forall z, w \in \mathbb{C}:|z+w| \leq|z|+|w|$. What can you say if $|z+w|=|z|+|w|$ ?
		\item Reverse triangle inequality: $\forall x, w \in \mathbb{C}:||z|-|w|| \leq|z-w|$.
	\end{enumerate}


\textbf{Answer}


	\clearpage

\section{Question 4}

\textbf{Convex sets}

A set $\Omega \subset \mathbb{R}^{n}$ is called convex if for all $x, y \in \Omega$ and for all $\lambda \in[0,1]$, the vector $\lambda x+(1-\lambda) y$ is also contained in $\Omega$.

\begin{enumerate}
	\item Show that for each radius $r \geq 0$ and each base point $p \in \mathbb{R}^{n}$, the open ball $B_{r}(p)=\left\{x \in \mathbb{R}^{n}\right.$ : $|x-p|<r\}$ is convex.
	\item Given two bounded convex sets $A, B \subset \mathbb{R}^{n}$, is the set $C:=\{2 \cdot a+b: a \in A, b \in B\}$ also convex? Either prove your claim, or give a counter example.
\end{enumerate}


\textbf{Answer}


	\clearpage
 \section{Question 5}
 
 \textbf{Metric spaces}
 
Show that the following are metric spaces:

\begin{enumerate}
\item $C^{0}([0,1])$ with the metric $d_{C^{0}}(f, g)=\sup _{x \in[0,1]}|f(x)-g(x)|$.
\item $\mathbb{R}^{2}$ with the French railway metric.
\end{enumerate}


\textbf{Answer}

	\clearpage

 \section{Question 6}

\textbf{Normed vector spaces}

Give an example of a metric spaces whose metric is not induced by a norm. In particular, not every metric space is a normed vector space.


\textbf{Answer}

	\clearpage
	
 \section{Question 7}

\textbf{Inner products}

For vectors $x=\left(x_{1}, \ldots, x_{n}\right), y=\left(y_{1}, \ldots, y_{n}\right) \in \mathbb{R}^{n}$, the inner product is defined by

$$
x \cdot y:=\sum_{i=1}^{n} x_{i} y_{i}:=x_{1} y_{1}+\cdots+x_{n} y_{n} .
$$

Let $n \geq 2$. Prove that for each $x \in \mathbb{R}^{n}$, there exists a $y \in \mathbb{R}^{n}$ with $y \neq 0$ such that $x \cdot y=0$. In this case we say that the vectors $x$ and $y$ are perpendicular to each other.

\textbf{Answer}


\vspace*{0.2\textheight}

\section{Question 8}

\textbf{Square roots}

Show that the number 3 has no square root in the rational numbers, i.e. the equation $p^{2}=3$ has no solution $p \in \mathbb{Q}$.


\textbf{Answer}

	\end{document}
