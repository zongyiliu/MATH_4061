\documentclass[letterpaper]{article} 
\usepackage[utf8]{inputenc}
\usepackage[T1]{fontenc}
\linespread{0.85}
\usepackage{amsmath}
\usepackage{amsfonts}
\usepackage{amssymb}
\usepackage{array}
\usepackage{booktabs}
\usepackage{hyperref}
\usepackage[version=4]{mhchem}
\usepackage{stmaryrd}
\usepackage[dvipsnames]{xcolor}
\colorlet{LightRubineRed}{RubineRed!70}
\colorlet{Mycolor1}{green!10!orange}
\definecolor{Mycolor2}{HTML}{00F9DE}
\usepackage{graphicx}
\usepackage{amsmath}
\usepackage{graphicx}
\usepackage{capt-of}
\usepackage{lipsum}
\usepackage{fancyvrb}
\usepackage{tabularx}
\usepackage{listings}
\usepackage[export]{adjustbox}
\graphicspath{ {./images/} }
\usepackage[utf8]{inputenc}
\usepackage[english]{babel}
\usepackage{float}
\usepackage{lipsum}
\usepackage{graphicx}
\usepackage{float}
\usepackage[margin=0.7in]{geometry}
\usepackage{amsmath}
\usepackage{graphicx}
\usepackage{capt-of}
\usepackage{tcolorbox}
\usepackage{lipsum}
\usepackage{graphicx}
\usepackage{float}
\usepackage{listings}
\usepackage{hyperref} 
\usepackage{xcolor} % For custom colors
\lstset{
	language=Python,                % Choose the language (e.g., Python, C, R)
	basicstyle=\ttfamily\small, % Font size and type
	keywordstyle=\color{blue},  % Keywords color
	commentstyle=\color{gray},  % Comments color
	stringstyle=\color{red},    % String color
	numbers=left,               % Line numbers
	numberstyle=\tiny\color{gray}, % Line number style
	stepnumber=1,               % Numbering step
	breaklines=true,            % Auto line break
	backgroundcolor=\color{black!5}, % Light gray background
	frame=single,               % Frame around the code
}
\usepackage{float}
\usepackage[]{amsthm} %lets us use \begin{proof}
	\usepackage[]{amssymb} %gives us the character \varnothing
	
	\title{Homework 4, MATH 4061}
	\author{Zongyi Liu}
	\date{Mon, Oct 6, 2025}
	\begin{document}
		\maketitle
		
		\section{Question 1}
		\textbf{Interval nesting principle}
		
		Show that the nterval nesting principle together with the Archimedean principle imply the completeness axiom.
		
		
		\textbf{Answer}
		
		\clearpage
		
		\section{Question 2}
		
		\textbf{Cantor set}
		
		We define iteratively $C_{0}:=[0,1]$, and
		
		$$
		\begin{aligned}
			C_{n} & :=\frac{1}{3}\left(C_{n-1} \cup\left(2+C_{n-1}\right)\right) \\
			& =\left\{x \in \mathbb{R}: x=\frac{1}{3} y \text { for some } y \in C_{n-1}\right\} \cup\left\{x \in \mathbb{R}: x=\frac{2}{3}+\frac{1}{3} y \text { for some } y \in C_{n-1}\right\} .
		\end{aligned}
		$$
		
		Moreover, we set
		
		$$
		\mathcal{C}:=\bigcap_{n=0}^{\infty} C_{n},
		$$
		
		and call $\mathcal{C}$ the Cantor set. What are the interior and the limit points of $\mathcal{C}$? Is $\mathcal{C}$ open or closed?
		
		\textbf{Answer}
		
		
		
		\clearpage
		
		\section{Question 3}
		
		\textbf{Separable spaces}
		
		Recall that a set $A$ is called countable if there exists an injection $f: A \rightarrow \mathbb{N}$. Let ( $X, d$ ) be a metric space. A subset $A \subseteq X$ is called dense in $X$ if every point of $X$ is a limit point of $A$, or a point of $A$. Moreover, $X$ is called separable if there exists a countable dense subset $A \subset X$. Show that $\mathbb{R}^{n}$ is separable. Can you find an example of a non-separable space?
		
		\textbf{Answer}
		
		
		\clearpage
		
		\section{Question 4}
		
		\textbf{Relative topology}
		
		
		Let ($X, d$) be a metric space and let $A \subseteq Y \subseteq X$. Show that $A$ is open relative to $Y$ if and only if $A=Y \cap B$ for some open set $B$ in $X$.
		
		\textbf{Answer}
		
		
		\clearpage
		\section{Question 5}
		
		\textbf{Connected spaces}
		
		Let ($X, d$) be a metric space. Show that $X$ is connected if and only if $\emptyset$ and $X$ are the only subsets of $X$ which are both closed and open.
		
		\textbf{Answer}
		
		\clearpage
		
		
		
		\section{Question 6}
		
		\textbf{Connectedness and unions}
		
		
		Let ( $X, d$ ) be a metric space and let $A, B \subseteq X$ be connected subspaces such that $A \cap B \neq \emptyset$. Show that $A \cup B$ is also connected.
		
		
		\textbf{Answer}
		
		\clearpage
		
		\section{Question 7}
		
		\textbf{Connected sets of $\mathbb{R}$}
		
		Show that the connected subsets of the real line $\mathbb{R}$ are the intervals (open, closed, half-open, halfclosed, open unbounded, closed unbounded), the empty set, and the real line itself.
		
		
		\textbf{Answer}
		
		
		\clearpage
		
		\section{Question 8}
		
		\textbf{Limit point compactness}
		
		Let ($X, d$) be a metric space and let $A \subseteq X$. Show that $A$ is compact if and only if $A$ is limit-point compact.
		
		\textbf{Answer}
		
		
	\end{document}
