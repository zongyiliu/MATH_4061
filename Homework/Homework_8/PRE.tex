\documentclass[10pt]{article}
\usepackage[utf8]{inputenc}
\usepackage[T1]{fontenc}
\usepackage{amsmath}
\usepackage{amsfonts}
\usepackage{amssymb}
\usepackage[version=4]{mhchem}
\usepackage{stmaryrd}
\usepackage{bbold}

\title{Introduction to Modern Analysis }

\author{}
\date{}


\begin{document}
\maketitle
\section*{Exercise Sheet 8}
Sven Hirsch\\
Columbia University, Fall 2025

Instructions: Complete all problems. Show all work for full credit. You may collaborate, but solutions must be written individually. Please turn in your solutions by $11 / 15 / 2025$.

\section*{Exercise 1: Differentiable functions}
Analyze which of the following functions are differentiable:

\begin{enumerate}
  \item The function $f: \mathbb{R} \rightarrow \mathbb{R}$ with $f(x)=|x|$.
  \item The function $g: \mathbb{R} \rightarrow \mathbb{R}$ with $g(x)=0$ for $x \leq 0$ and $g(x)=x^{2} \sin \left(\frac{1}{x}\right)$ for $x>0$.
\end{enumerate}

\section*{Exercise 2: Constant functions}
Let $a>1$. Suppose that $f: \mathbb{R} \rightarrow \mathbb{R}$ satisfies $|f(x)-f(y)| \leq|x-y|^{a}$ for all $x, y \in \mathbb{R}$. Show that $f$ is constant.

\section*{Exercise 3: Monotone functions}
Let $f:[0,1] \rightarrow \mathbb{R}$ be a differentiable function. Show that $f$ monotonically increasing if and only if $f^{\prime}(x) \geq 0$ for all $x \in[0,1]$.

\section*{Exercise 4: Derivatives}
Let $f, g, h: \mathbb{R} \rightarrow \mathbb{R}$ be differentiable functions. Moreover, assume that $g(x) \neq 0$ for all $x \in \mathbb{R}$, and that $h$ is bijective with inverse function $h^{-1}$. Are $\frac{f}{g}$ and $h^{-1}$ also differentiable?

\section*{Exercise 5: Lipschitz functions}
Let $f: \mathbb{R} \rightarrow \mathbb{R}$ be a differentiable functions such that $f^{\prime}$ is bounded. Prove that $f$ is Lipschitz continuous.

\section*{Exercise 6: Power rule}
Let $r \in \mathbb{R}$. Compute the derivative of the function $f: \mathbb{R}_{>0} \rightarrow \mathbb{R}$ given by $f(x)=x^{r}$.

\section*{Exercise 7: Newton's method}
Complete Exercise 25 of Chapter 5 in Rudin's book.

\section*{Exercise 8: Ordinary differential equations}
Complete Exercise 27 of Chapter 5 in Rudin's book.\\
Bonus questions: Under the same assumptions as Exercise 27, prove that solutions to the initial value problem exist. More precisely, given $a, c \in \mathbb{R}$, show that there exists $\varepsilon>0$ and a unique function $f:[a, a+\varepsilon] \rightarrow \mathbb{R}$ such that $f(a)=c$ and $f^{\prime}(x)=\phi(x, f(x))$ for $x \in[a, a+\varepsilon]$.

Hint: Define iteratively a sequence of functions by setting $y_{0}=c$ and

$$
y_{i+1}(t)=c+\int_{a}^{t} \phi\left(x, y_{i}(x)\right) d x
$$

and use the Banach fixed point theorem. You do not need to show that the underlying function space is complete.

Sven Hirsch\\
Columbia University\\
Fall 2025


\end{document}
