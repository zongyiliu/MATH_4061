\documentclass[10pt]{article}
\usepackage[utf8]{inputenc}
\usepackage[T1]{fontenc}
\usepackage{amsmath}
\usepackage{amsfonts}
\usepackage{amssymb}
\usepackage[version=4]{mhchem}
\usepackage{stmaryrd}
\usepackage{bbold}

\title{Introduction to Modern Analysis }

\author{Exercise Sheet 10\\
Sven Hirsch\\
Columbia University, Fall 2025}
\date{}


\begin{document}
\maketitle
Instructions: Complete all problems. Show all work for full credit. You may collaborate, but solutions must be written individually. Please turn in your solutions by 12/14/2025.

\section*{Exercise 1: Uniform convergence}
Let $f_{n}, g_{n}:(0,1) \rightarrow \mathbb{R}$ be two sequences of continuous functions such that both $\left(f_{n}\right)_{n \in \mathbb{N}}$ and $\left(g_{n}\right)_{n \in \mathbb{N}}$ converge uniformly for $n \rightarrow \infty$.\\
a) In case $\left(f_{n}\right)_{n \in \mathbb{N}}$ and $\left(g_{n}\right)_{n \in \mathbb{N}}$ are sequences of bounded functions, show that also $\left(f_{n} g_{n}\right)_{n \in \mathbb{N}}$ converges uniformly for $n \rightarrow \infty$.\\
b) Show that it is necessary to assume boundedness by giving an example where $\left(f_{n} g_{n}\right)_{n \in \mathbb{N}}$ does not converge uniformly.

\section*{Exercise 2: Chain rule}
Let $f:(a, b) \rightarrow \mathbb{R}$ and $g:(c, d) \rightarrow \mathbb{R}$ be functions such that $f((a, b)) \subset(c, d)$. Suppose that $f$ is differentiable at $x_{0} \in(a, b)$ and that $g$ is differentiable at $f\left(x_{0}\right)$. Prove that

$$
(g \circ f)^{\prime}\left(x_{0}\right)=g^{\prime}\left(f\left(x_{0}\right)\right) f^{\prime}\left(x_{0}\right) .
$$

\section*{Exercise 3: Integration by parts}
Let $f, g:[a, b] \rightarrow \mathbb{R}$ be differentiable functions whose derivatives are Riemann integrable. Show that

$$
\int_{a}^{b} f(x) g^{\prime}(x) d x=f(b) g(b)-f(a) g(a)-\int_{a}^{b} f^{\prime}(x) g(x) d x
$$

\section*{Exercise 4: Integration by Substitution}
Let $\varphi:[\alpha, \beta] \rightarrow[a, b]$ be a continuously differentiable, strictly monotone function, and let $f: [a, b] \rightarrow \mathbb{R}$ be continuous. Prove the substitution rule

$$
\int_{\varphi(\alpha)}^{\varphi(\beta)} f(u) d u=\int_{\alpha}^{\beta} f(\varphi(t)) \varphi^{\prime}(t) d t
$$

\section*{Exercise 5: Completeness of bounded continuous functions}
Let $C_{b}([0,1])$ be the space of bounded continuous functions on the interval $[0,1]$. Define the norm

$$
\|f\|_{L^{1}}:=\int_{0}^{1}|f(x)| d x
$$

and the corresponding metric $d_{L^{1}}(f, g)=\|f-g\|_{L^{1}}$. Is the metric space $\left(C_{b}([0,1]), d_{L^{1}}\right)$ complete?

\section*{Exercised 6: Convergence and differentiation}
Let $f_{n}:[a, b] \rightarrow \mathbb{R}$ be a sequence of differentiable functions. Show that if $f_{n}$ converges uniformly to $f$, and if the derivatives $f_{n}^{\prime}$ converge uniformly to $g$, then $f$ is differentiable and $f^{\prime}=g$. By giving an example, show that the above assumptions are necessary. What does this imply for series?

\section*{Exercise 7: Differentiation of Power Series}
Let $\left(c_{n}\right)_{n \in \mathbb{N}}$ be a sequence of complex numbers, and consider the power series

$$
f(x)=\sum_{n=0}^{\infty} c_{n} x^{n}
$$

with radius of convergence $R>0$. Show that $f$ is differentiable on ( $-R, R$ ) and that its derivative is given by

$$
f^{\prime}(x)=\sum_{n=1}^{\infty} n c_{n} x^{n-1}
$$

and that the differentiated series also has radius of convergence $R$.

\section*{Exercise 8: Uniform boundedness}
Let $(X, d)$ be a compact metric space, and let $f_{n}: X \rightarrow \mathbb{R}, n \in \mathbb{N}$, be a pointwise bounded and uniformly equicontinuous sequence of functions, i.e.

$$
\forall x \in X: \exists M \in \mathbb{R}: \forall n \in \mathbb{N}:\left|f_{n}(x)\right| \leq M
$$

and

$$
\forall \varepsilon>0: \exists \delta>0: \forall x, y \in X: \forall n \in \mathbb{N}:\left(d_{X}(x, y) \leq \delta \Longrightarrow\left|f_{n}(x)-f_{n}(y)\right| \leq \varepsilon\right)
$$

Show that $\left(f_{n}\right)_{n \in \mathbb{N}}$ is also uniformly bounded, i.e.

$$
\exists C \in \mathbb{R}: \forall x \in X: \forall n \in \mathbb{N}:\left|f_{n}(x)\right| \leq C
$$

Sven Hirsch\\
Columbia University\\
Fall 2025


\end{document}
