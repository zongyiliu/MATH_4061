\documentclass[letterpaper]{article} 
\usepackage[utf8]{inputenc}
\usepackage[T1]{fontenc}
\usepackage{amsmath}
\usepackage{amsfonts}
\usepackage{amssymb}
\usepackage{array}
\usepackage{booktabs}
\usepackage{hyperref}
\usepackage[version=4]{mhchem}
\usepackage{stmaryrd}
\usepackage[dvipsnames]{xcolor}
\colorlet{LightRubineRed}{RubineRed!70}
\colorlet{Mycolor1}{green!10!orange}
\definecolor{Mycolor2}{HTML}{00F9DE}
\usepackage{graphicx}
\usepackage{amsmath}
\usepackage{graphicx}
\usepackage{capt-of}
\usepackage{lipsum}
\usepackage{fancyvrb}
\usepackage{tabularx}
\usepackage{listings}
\usepackage[export]{adjustbox}
\graphicspath{ {./images/} }
\usepackage[utf8]{inputenc}
\usepackage[english]{babel}
\usepackage{float}
\usepackage{lipsum}
\usepackage{graphicx}
\usepackage{float}
\usepackage[margin=0.7in]{geometry}
\usepackage{amsmath}
\usepackage{graphicx}
\usepackage{capt-of}
\usepackage{tcolorbox}
\usepackage{lipsum}
\usepackage{graphicx}
\usepackage{float}
\usepackage{listings}
\usepackage{hyperref} 
\usepackage{xcolor} % For custom colors
\lstset{
	language=Python,                % Choose the language (e.g., Python, C, R)
	basicstyle=\ttfamily\small, % Font size and type
	keywordstyle=\color{blue},  % Keywords color
	commentstyle=\color{gray},  % Comments color
	stringstyle=\color{red},    % String color
	numbers=left,               % Line numbers
	numberstyle=\tiny\color{gray}, % Line number style
	stepnumber=1,               % Numbering step
	breaklines=true,            % Auto line break
	backgroundcolor=\color{black!5}, % Light gray background
	frame=single,               % Frame around the code
}
\usepackage{float}
\usepackage[]{amsthm} %lets us use \begin{proof}
\usepackage[]{amssymb} %gives us the character \varnothing

	\title{Homework 6, MATH 4061}
	\author{Zongyi Liu}
	\date{Fri, Oct 17, 2025}
	\begin{document}
		\maketitle
		
		\section{Question 1}
		\textbf{Comparison test}
		
		Let $\left\{a_{i}\right\}_{i=0}^{\infty},\left\{b_{i}\right\}_{i=0}^{\infty}$ and $\left\{c_{i}\right\}_{i=0}^{\infty}$ be sequences of real numbers and let $i_{0} \in \mathbb{N}$. Show that the following holds:
		
		\begin{enumerate}
			\item If $\left|a_{i}\right| \leq b_{i}$ for all $i \geq i_{0}$ and if $\sum_{i=0}^{\infty} b_{i}$ converges, then $\sum_{i=0}^{\infty} a_{i}$ converges absolutely.
			\item If $a_{i} \geq c_{i} \geq 0$ for $i \geq i_{0}$ and if $\sum_{i=0}^{\infty} c_{i}$ diverges, then $\sum_{i=0}^{\infty} a_{i}$ also diverges.
		\end{enumerate}
	
		
		\textbf{Answer}
		
		\clearpage
	
	
	\section{Question 2}
	\textbf{Root test}
	
	Let $\left(x_{n}\right)_{n \in \mathbb{N}}$ be a sequence in $\mathbb{R}$. Show that the series $\sum_{n=0}^{\infty} a_{n}$
	
	\begin{itemize}
		\item converges absolutely if $\limsup _{n \rightarrow \infty}\left|a_{n}\right|^{\frac{1}{n}}<1$,
		\item diverges if $\lim \sup _{n \rightarrow \infty}\left|a_{n}\right|^{\frac{1}{n}}>1$.
	\end{itemize}

\textbf{Answer}

\clearpage


	\section{Question 3}
\textbf{Series and monotone sequences}

Let $\sum_{n=0}^{\infty} a_{n}$ be a convergent series and let $b_{n}$ be a monotonically increasing and bounded sequence. Prove that $\sum_{n=0}^{\infty} a_{n} b_{n}$ also converges.

\textbf{Answer}

\clearpage


\section{Question 4}
\textbf{Riemann rearrangement theorem}

Let $\sum_{n=1}^{\infty} a_{n}$ be a series of real numbers. A rearrangement of this series is a new series

$$
\sum_{n=1}^{\infty} b_{n},
$$

where $b_{n}=a_{\sigma(n)}$ for some bijection $\sigma: \mathbb{N} \rightarrow \mathbb{N}$. The bijection $\sigma$ reorders the terms of the original series without omitting or repeating any of them.

Let $c \in \mathbb{R}$ and suppose that $\sum_{n=1}^{\infty} a_{n}$ converges conditionally, i.e. $\sum_{n=1}^{\infty} a_{n}$ converges, but $\sum_{n=1}^{\infty}\left|a_{n}\right|$ diverges. Show that there exists a rearrangement of $\sum_{n=1}^{\infty} a_{n}$ which converges to $c$.

\textbf{Answer}

\clearpage

	\section{Question 5}
	
\textbf{Leibniz criterion}

Suppose that $\left\{a_{i}\right\}_{i=0}^{\infty}$ is a sequence of non-negative real numbers which is decreasing and which converges to zero. Show that

$$
\sum_{i=0}^{\infty}(-1)^{i} a_{i}
$$

converges. Does this series also converge absolutely?

\textbf{Answer}

\clearpage

\section{Question 6}

\textbf{Radius of convergence}

Let $z \in \mathbb{C}$ and let $\left\{a_{i}\right\}_{i=0}^{\infty}$ be a sequence of complex numbers. Given the power series $\sum_{i=0}^{\infty} a_{i} z^{i}$, recall that the radius of convergence $R$ is defined by $R=\frac{1}{\rho}$ where $\rho=\limsup _{i \rightarrow \infty}\left|a_{i}\right|^{\frac{1}{i}}$. Show that the power series converges for $|z|<R$ and diverges for $|z|>R$.

\textbf{Answer}

\clearpage

\section{Question 7}

\textbf{Euler's identity}

We define for $z \in \mathbb{C}$ the exponential, the sine, and the cosine function by

$$
\exp (z)=\sum_{n=0}^{\infty} \frac{z^{n}}{n!}, \quad \sin (z)=\sum_{n=0}^{\infty} \frac{(-1)^{n}}{(2 n+1)!} z^{2 n+1}, \quad \cos (z)=\sum_{n=0}^{\infty} \frac{(-1)^{n}}{(2 n)!} z^{2 n}
$$

For which values of $z \in \mathbb{C}$ do these series converge? Moreover, show that

$$
\exp (i z)=\cos (z)+i \sin (z)
$$


\textbf{Answer}

\clearpage

\section{Question 8}

\textbf{Continuous functions}

Analyze where the following functions are continuous:

\begin{enumerate}
\item $f: \mathbb{R} \rightarrow \mathbb{R}$ given by $f(x)=|x|$.
\item $g: \mathbb{R} \rightarrow \mathbb{R}$ defined by $g(x)=0$ for $x \in \mathbb{Q}$ and $g(x)=x$ for $x \in \mathbb{R} \backslash \mathbb{Q}$.
\item $h: \mathbb{R} \backslash\{0\} \rightarrow \mathbb{R}$ defined by $h(x)=\frac{|x|}{x}$.
\item The distance function $d: X \times X \rightarrow \mathbb{R}$ where ( $X, d$ ) is any metric space.
\item The Cantor function $c:[0,1] \rightarrow[0,1]$.
\item The popcorn function $p:[0: 1] \rightarrow[0,1]$
\end{enumerate}

\textbf{Answer}

\clearpage
	
	
	\end{document}
