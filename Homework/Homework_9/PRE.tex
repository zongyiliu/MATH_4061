\documentclass[10pt]{article}
\usepackage[utf8]{inputenc}
\usepackage[T1]{fontenc}
\usepackage{amsmath}
\usepackage{amsfonts}
\usepackage{amssymb}
\usepackage[version=4]{mhchem}
\usepackage{stmaryrd}
\usepackage{bbold}

\title{Introduction to Modern Analysis }

\author{}
\date{}


\begin{document}
\maketitle
\section*{Exercise Sheet 9}
Sven Hirsch\\
Columbia University, Fall 2025

Instructions: Complete all problems. Show all work for full credit. You may collaborate, but solutions must be written individually. Please turn in your solutions by $11 / 30 / 2025$.

\section*{Exercise 1: Integration via Darboux sums}
Let $c>0$. Evaluate the following integral $\int_{0}^{c} x d x$ using Darboux sums, and not using the fundamental theorem of calculus.

\section*{Exercise 2: Properties of the integral}
Let $f, g:[a, b] \rightarrow \mathbb{R}$ be Darboux integrable functions and let $c \in \mathbb{R}$. Show that the integral has the following properties:

\begin{enumerate}
  \item $\int_{a}^{b}(f+g)=\int_{a}^{b} f+\int_{a}^{b} g$.
  \item $\int_{a}^{b}(c f)=c \int_{a}^{b} f$.
  \item $\int_{a}^{b} f \leq \int_{a}^{b} g$ if we additionally assume $f \leq g$.
  \item $\left|\int_{a}^{b} f\right| \leq \int_{a}^{b}|f|$.
\end{enumerate}

\section*{Exercise 3: Composition of integrable functions}
Let $f, g:[0,1] \rightarrow[0,1]$ be Darboux integrable functions. Is the composition $f \circ g$ also Darboux integrable?

\section*{Exercise 4: Characterization of integrability}
Let $f$ be a bounded function on the interval $[a, b]$. Show that $f$ is Darboux integrable if and only if the set of discontinuities is of measure zero.

Here we say a set $A \subset \mathbb{R}$ is of measure zero if for every $\varepsilon>0$ there exists a family of open intervals $\left\{O_{n}\right\}_{n \in \mathbb{N}}$ such that $A \subseteq \bigcup_{n \in \mathbb{N}} O_{n}$ and such that the total length of the intervals is less than $\varepsilon$.

\section*{Exercise 5: More integrals}
Compute the following integrals.

\begin{enumerate}
  \item $\int_{0}^{1} x^{2} e^{x} d x$,
  \item $\int_{-1}^{1} \sqrt{1-x^{2}} d x$.
\end{enumerate}

\section*{Exercise 6: Hölder's inequality}
Let $f, g:[a, b] \rightarrow \mathbb{R}$ be two Darboux integrable functions and let $p, q$ be two positive real numbers such that $\frac{1}{p}+\frac{1}{q}=1$. Show that

$$
\left|\int_{a}^{b} f g d x\right| \leq\left(\int_{a}^{b}|f|^{p} d x\right)^{\frac{1}{p}}\left(\int_{a}^{b}|g|^{q} d x\right)^{\frac{1}{q}}
$$

\section*{Exercise 7: Length of curves}
Let $f:[a, b] \rightarrow \mathbb{R}$ be differentiable. We define the length of the graph of $f$ via

$$
\mathcal{L}(f)=\sup _{P \in \mathcal{P}} \sum_{k=1}^{n} \sqrt{\left(x_{k}-x_{k-1}\right)^{2}+\left(f\left(x_{k}\right)-f\left(x_{k-1}\right)\right)^{2}}
$$

where $\mathcal{P}$ is the set of all partitions $P=\left\{x_{0}, \ldots, x_{n}\right\}$ of the interval $[a, b]$. Show that

$$
\mathcal{L}(f)=\int_{a}^{b} \sqrt{1+f^{\prime}(x)^{2}} d x
$$

Use the above formula to show that the circumference of a circle of radius $r$ equals $2 \pi r$.

\section*{Exercise 8: Fundamental lemma of the calculus of variations (4 points)}
Let $f:[a, b] \rightarrow \mathbb{R}$ be a continuous function such that

$$
\int_{a}^{b} f(x) \phi(x) d x=0
$$

for every continuous function $\phi:[a, b] \rightarrow \mathbb{R}$. Show that $f(x)=0$ for all $x \in[a, b]$.

Sven Hirsch\\
Columbia University\\
Fall 2025


\end{document}
