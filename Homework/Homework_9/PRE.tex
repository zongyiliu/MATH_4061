\documentclass[letterpaper]{article} 
\usepackage[utf8]{inputenc}
\usepackage[T1]{fontenc}
\usepackage{amsmath}
\usepackage{amsfonts}
\usepackage{amssymb}
\usepackage{array}
\usepackage{booktabs}
\usepackage{hyperref}
\usepackage[version=4]{mhchem}
\usepackage{stmaryrd}
\usepackage[dvipsnames]{xcolor}
\colorlet{LightRubineRed}{RubineRed!70}
\colorlet{Mycolor1}{green!10!orange}
\definecolor{Mycolor2}{HTML}{00F9DE}
\usepackage{graphicx}
\usepackage{amsmath}
\usepackage{graphicx}
\usepackage{capt-of}
\usepackage{lipsum}
\usepackage{fancyvrb}
\usepackage{tabularx}
\usepackage{listings}
\usepackage[export]{adjustbox}
\graphicspath{ {./images/} }
\usepackage[utf8]{inputenc}
\usepackage[english]{babel}
\usepackage{float}
\usepackage{lipsum}
\usepackage{graphicx}
\usepackage{float}
\usepackage[margin=0.7in]{geometry}
\usepackage{amsmath}
\usepackage{graphicx}
\usepackage{capt-of}
\usepackage{tcolorbox}
\usepackage{lipsum}
\usepackage{graphicx}
\usepackage{float}
\usepackage{listings}
\usepackage{hyperref} 
\usepackage{xcolor} % For custom colors
\lstset{
	language=Python,                % Choose the language (e.g., Python, C, R)
	basicstyle=\ttfamily\small, % Font size and type
	keywordstyle=\color{blue},  % Keywords color
	commentstyle=\color{gray},  % Comments color
	stringstyle=\color{red},    % String color
	numbers=left,               % Line numbers
	numberstyle=\tiny\color{gray}, % Line number style
	stepnumber=1,               % Numbering step
	breaklines=true,            % Auto line break
	backgroundcolor=\color{black!5}, % Light gray background
	frame=single,               % Frame around the code
}
\usepackage{float}
\usepackage[]{amsthm} %lets us use \begin{proof}
	\usepackage[]{amssymb} %gives us the character \varnothing
	
	\title{Homework 9, MATH 4061}
	\author{Zongyi Liu}
	\date{Fri, Nov 21, 2025}
	\begin{document}
		\maketitle
		
		\section{Question 1}
		\textbf{Integration via Darboux sums}
		
		
		Let $c>0$. Evaluate the following integral $\int_{0}^{c} x d x$ using Darboux sums, and not using the fundamental theorem of calculus.
		
		
		\textbf{Answer}
		
		
		\clearpage
		
		
		\section{Question 2}
		\textbf{Properties of the integral}
		
		Let $f, g:[a, b] \rightarrow \mathbb{R}$ be Darboux integrable functions and let $c \in \mathbb{R}$. Show that the integral has the following properties:
		
		\begin{enumerate}
			\item $\int_{a}^{b}(f+g)=\int_{a}^{b} f+\int_{a}^{b} g$.
			\item $\int_{a}^{b}(c f)=c \int_{a}^{b} f$.
			\item $\int_{a}^{b} f \leq \int_{a}^{b} g$ if we additionally assume $f \leq g$.
			\item $\left|\int_{a}^{b} f\right| \leq \int_{a}^{b}|f|$.
		\end{enumerate}
		
		\textbf{Answer}
		
		\clearpage
		
		
		\section{Question 3}
		\textbf{Composition of integrable functions}
		
		Let $f, g:[0,1] \rightarrow[0,1]$ be Darboux integrable functions. Is the composition $f \circ g$ also Darboux integrable?
		
		
		\textbf{Answer}
		
		\clearpage
		
		
		\section{Question 4}
		\textbf{Characterization of integrability}
		
		
		Let $f$ be a bounded function on the interval $[a, b]$. Show that $f$ is Darboux integrable if and only if the set of discontinuities is of measure zero.
		
		Here we say a set $A \subset \mathbb{R}$ is of measure zero if for every $\varepsilon>0$ there exists a family of open intervals $\left\{O_{n}\right\}_{n \in \mathbb{N}}$ such that $A \subseteq \bigcup_{n \in \mathbb{N}} O_{n}$ and such that the total length of the intervals is less than $\varepsilon$.
		
		\textbf{Answer}
		
		
		
		\clearpage
		
		\section{Question 5}
		
		\textbf{More integrals}
		
		Compute the following integrals.
		
		\begin{enumerate}
			\item $\int_{0}^{1} x^{2} e^{x} d x$,
			\item $\int_{-1}^{1} \sqrt{1-x^{2}} d x$.
		\end{enumerate}
		
		\textbf{Answer}
		
		\clearpage
		
		\section{Question 6}
		
		\textbf{Hölder's inequality}
		
		Let $f, g:[a, b] \rightarrow \mathbb{R}$ be two Darboux integrable functions and let $p, q$ be two positive real numbers such that $\frac{1}{p}+\frac{1}{q}=1$. Show that
		
		$$
		\left|\int_{a}^{b} f g d x\right| \leq\left(\int_{a}^{b}|f|^{p} d x\right)^{\frac{1}{p}}\left(\int_{a}^{b}|g|^{q} d x\right)^{\frac{1}{q}}
		$$
		
		
		\textbf{Answer}
		

		
		\clearpage
		
		\section{Question 7}
		
		\textbf{Length of curves}
		
		Let $f:[a, b] \rightarrow \mathbb{R}$ be differentiable. We define the length of the graph of $f$ via
		
		$$
		\mathcal{L}(f)=\sup _{P \in \mathcal{P}} \sum_{k=1}^{n} \sqrt{\left(x_{k}-x_{k-1}\right)^{2}+\left(f\left(x_{k}\right)-f\left(x_{k-1}\right)\right)^{2}}
		$$
		
		where $\mathcal{P}$ is the set of all partitions $P=\left\{x_{0}, \ldots, x_{n}\right\}$ of the interval $[a, b]$. Show that
		
		$$
		\mathcal{L}(f)=\int_{a}^{b} \sqrt{1+f^{\prime}(x)^{2}} d x
		$$
		
		Use the above formula to show that the circumference of a circle of radius $r$ equals $2 \pi r$.
		
		\textbf{Answer}
		
		
		\clearpage
		
		\section{Question 8}
		
		
		\textbf{Fundamental lemma of the calculus of variations}
		Let $f:[a, b] \rightarrow \mathbb{R}$ be a continuous function such that
		
		$$
		\int_{a}^{b} f(x) \phi(x) d x=0
		$$
		
		for every continuous function $\phi:[a, b] \rightarrow \mathbb{R}$. Show that $f(x)=0$ for all $x \in[a, b]$.
		
		
		\textbf{Answer}
		
		
		\clearpage
		
		
	\end{document}
