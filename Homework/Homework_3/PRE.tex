\documentclass[letterpaper]{article} 
\linespread{0.85}
\usepackage[utf8]{inputenc}
\usepackage[T1]{fontenc}
\usepackage{amsmath}
\usepackage{amsfonts}
\usepackage{amssymb}
\usepackage{array}
\usepackage{booktabs}
\usepackage{hyperref}
\usepackage[version=4]{mhchem}
\usepackage{stmaryrd}
\usepackage[dvipsnames]{xcolor}
\colorlet{LightRubineRed}{RubineRed!70}
\colorlet{Mycolor1}{green!10!orange}
\definecolor{Mycolor2}{HTML}{00F9DE}
\usepackage{graphicx}
\usepackage{amsmath}
\usepackage{graphicx}
\usepackage{capt-of}
\usepackage{lipsum}
\usepackage{fancyvrb}
\usepackage{tabularx}
\usepackage{listings}
\usepackage[export]{adjustbox}
\graphicspath{ {./images/} }
\usepackage[utf8]{inputenc}
\usepackage[english]{babel}
\usepackage{float}
\usepackage{lipsum}
\usepackage{graphicx}
\usepackage{float}
\usepackage[margin=0.7in]{geometry}
\usepackage{amsmath}
\usepackage{graphicx}
\usepackage{capt-of}
\usepackage{tcolorbox}
\usepackage{lipsum}
\usepackage{graphicx}
\usepackage{float}
\usepackage{listings}
\usepackage{hyperref} 
\usepackage{xcolor} % For custom colors
\lstset{
	language=Python,                % Choose the language (e.g., Python, C, R)
	basicstyle=\ttfamily\small, % Font size and type
	keywordstyle=\color{blue},  % Keywords color
	commentstyle=\color{gray},  % Comments color
	stringstyle=\color{red},    % String color
	numbers=left,               % Line numbers
	numberstyle=\tiny\color{gray}, % Line number style
	stepnumber=1,               % Numbering step
	breaklines=true,            % Auto line break
	backgroundcolor=\color{black!5}, % Light gray background
	frame=single,               % Frame around the code
}
\usepackage{float}
\usepackage[]{amsthm} %lets us use \begin{proof}
\usepackage[]{amssymb} %gives us the character \varnothing

	\title{Homework 3, MATH 4061}
	\author{Zongyi Liu}
	\date{Sat, Sept 27, 2025}
	\begin{document}
		\maketitle
		
		\section{Question 1}
		\textbf{ Isolated points}
		
		Let ($X, d$) be a metric space and let $A \subseteq X$. Suppose that $A$ is bounded and that every $a \in A$ is an isolated point. Prove or disprove whether $A$ must be a finite set, i.e. $A=\left\{a_{1}, \ldots, a_{n}\right\}$.


		\textbf{Answer}
		
		
		\clearpage
		
		\section{Question 2}
		
		\textbf{Hilbert's Hotel}
		
		
		A set $X$ is called countable if there exists a bijection of $X$ onto a subset of the natural numbers $\mathbb{N}$. A set $X$ is called uncountable if it is not countable.\\
		(a) Show that the set $\mathbb{N} \cup\{-1\}$ is countable.\\
		(b) Show that the integer numbers $\mathbb{Z}$ are countable.\\
		(c) Show that the rational numbers $\mathbb{Q}$ are countable.
		
	\textbf{Answer}
	
			\clearpage
	
	\section{Question 3}
	
	\textbf{Open and closed sets}
	
     Let $a<b \in \mathbb{R}, \varepsilon>0$, and let $A \subset B_{1}(0)$ be a closed non-empty subset of $\mathbb{R}^{2}$. Analyze whether the following sets are open or closed.\\
 (a) The interval $[a, b)$ as a subset of $\mathbb{R}$.\\
(b) The set $(\mathbb{N} \times\{0\}) \cup B_{1}(0)$ as a subset of $\mathbb{R}^{2}$.\\
(c) The set $A_{\varepsilon}:=\left\{x \in \mathbb{R}^{2}: \inf _{a \in A} d(x, a)<\varepsilon\right\}$ where $d$ is the Euclidean distance function on $\mathbb{R}^{2}$.\\
(d) The set $\{x\} \subseteq X$ where $X$ is an arbitrary set equipped with the discrete metric and $x \in X$.

\textbf{Answer}
\clearpage
\section{Question 4}

\textbf{Supremum and closed sets}

Let $A \subseteq \mathbb{R}$ be a bounded set and suppose that $A$ is closed. Show that $\sup A \in A$. What happens if $A$ is assumed to be open?

\textbf{Answer}

	\clearpage
 \section{Question 5}
 
 \textbf{Closure, interior and boundary}
 
 
 Let ($X, d$) be a metric space and let $A, B \subseteq X$ be non-empty subsets. Show that the following properties hold:\\
 (a) If $A \subseteq B$, then $\bar{A} \subseteq \bar{B}$.\\
 (b) The closure $\bar{A}$ is a closed subset of $X$.\\
 (c) The set $A \subseteq X$ is closed if and only if $A=\bar{A}$.\\
 (d) The set $A \subseteq X$ is open if and only if $A=A^{\circ}$.\\
 (e) If $X=\mathbb{R}$, and if $A \subset X$ is bounded from above, show that $\sup (A) \in \bar{A}$.\\
 (f) We have $\partial(\partial A) \subseteq \partial A$.\\
 (g) We have $\partial A=\bar{A} \cap \overline{A^{C}}$.

\textbf{Answer}


\clearpage



 \section{Question 6}

\textbf{Open covers}

Find an open cover of the open ball $B_{1}(0) \subseteq \mathbb{R}^{7}$ which has no finite subcover. Explain your answer.

\textbf{Answer}

	\clearpage
	
 \section{Question 7}

\textbf{Compactness}

Show that the set

$$
A=\{0\} \cup\left\{\frac{1}{n}: n \in \mathbb{N}, n>0\right\} \subseteq \mathbb{R}
$$

is compact.

\textbf{Answer}

	\clearpage

\section{Question 8}

\textbf{Diameter of the closure}

Let ( $X, d$ ) be a metric space and let $A \subset X$ be a bounded subset. Show that the diameter satisfies

$$
\operatorname{diam}(A)=\operatorname{diam}(\bar{A})
$$

\textbf{Answer}

	\end{document}
